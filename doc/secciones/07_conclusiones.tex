\chapter{Conclusiones y futuro del proyecto}

En este capítulo se verá si realmente se han cumplido los objetivos expuestos al comienzo del 
proyecto, en el caso de que lo hayan hecho, ver si la solución realmente ha sido óptima y los que aún no se hayan 
cumplido, como lo podrían hacer resumidamente. Y tratar alguna de las decisiones futuras del proyecto que estaría bien 
vislumbrar.

Al comienzo del proyecto se veía la necesidad del usuario por exponer su opinión referente al contenido consumido, en 
este aspecto se ha creado un sistema para la creación de reseñas fundadas, que de manera sencilla construye una reseña 
consistente describiendo aspectos fundamentales de cualquier película. También se mostraba el problema de indecisión 
entre la gente respecto a decidir que contenido cinematográfico iba a ver, debido a la cantidad de contenido existente, 
sin ninguna opción que conectara películas ya vistas con otras que les podían interesar. Por eso aparece en uno de los 
objetivos crear un algoritmo que sea capaz de recomendar a los usuarios contenidos según su consumo. Este es uno de los 
objetivos que se cree cumplido tras llegar al final del camino, ya que a través de las reseñas las recomendaciones le 
llueven al usuario, siempre conectadas de alguna forma con las películas ya consumidas y reseñadas. Otro de los 
objetivos que se cree cumplido es el de la creación de una aplicación para que el algoritmo de recomendaciones pueda ser 
consumido por los usuarios, además al crear una API indirectamente se puede apreciar una pequeña aportación para 
aquellos usuarios que quieran hacer uso del algoritmo en sus propias versiones de este proyecto, por lo que también 
sería consumido por desarrolladores.

Por otro lado, uno de los objetivos también era la creación de un algoritmo de fiabilidad de las reseñas. 
Lamentablemente con la solución actual no se ha podido lograr este objetivo, aunque un futuro próximo al proyecto podría 
conseguirlo. Ya que en la solución actual se ha podido satisfacer la historia del aficionado, la historia del cinéfilo 
tendría la clave para cumplir con el objetivo de lograr unas reseñas fiables, a través de distintos hitos, se obtendrían 
diversos PMV que terminaran cumpliendo con dicho objetivo y satisfaciendo las necesidades del cinéfilo. Es un objetivo 
que puede plantear varios problemas, pero que puede obtenerse a través de una solución no muy complicada, como sería el 
posible uso de un sistema de reputación como el existente en Stack Overflow. Permitiendo valorar al usuario cada reseña 
por el peso de la reputación del usuario que la haya escrito. Pero para obtener dicha solución, en este caso tendría que 
haberse solucionado el tema de la persistencia, ya que en la solución actual no ha sido necesaria, pero en esta 
situación para la integridad de los datos de reputación sería imprescindible.

Ya se discutiría que sistema de persistencia se usaría, relacional o no relacional, tratándose de un PMV en primera 
estancia y sin entrar en detalle se podría optar por una opción no relacional, ya que sería más flexible para el esquema 
pudiendo guardar las estructuras que se quiera mientras este más o menos bien organizado y la conversión para la 
respuesta de la API también podrían ser instantáneas siendo más sencilla su implementación. Además, ofrecen un gran 
desempeño para cargas de trabajo como realizar lecturas o escrituras rápidas y de un gran volumen de datos, gracias a 
poseer menos sobrecarga en la gestión de relaciones.

Aunque un sistema relacionar tampoco sería difícil de implantar aportando gran consistencia a los datos, permitiendo un 
mayor control sobre los datos y asegurando su consistencia. Pudiendo incluso sustituir el algoritmo de las 
recomendaciones por un conjunto de consultas complejas.

En un futuro se podría ver la elección de cualquiera de los dos sistemas, ya que ambos podrían ser implantados, la 
elección radicara en las necesidades del proyecto en ese momento, respecto a la escalabilidad de este, vertical u 
horizontal siendo un sistema no relacional o uno relacional la elección respectivamente.

