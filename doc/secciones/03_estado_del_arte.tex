\chapter{Estado del arte}

Es este capítulo se va a contemplar el mercado en el que se encuentra situada la solución software 
planteada en el capítulo anterior. Por lo que habrá que ver las distintas opciones que existen en el 
mercado actualmente y que serán competencia de la solución que se va a desarrollar. Una descripción 
detallada de estas aplicaciones o páginas web destinadas a las reseñas cinematográficas más populares 
en la actualidad y las características que las hacen destacar.

Tras ello se realizará una comparación entre ellas en la que se destaquen sus semejanzas y 
diferencias, y se analicen las ventajas e inconvenientes de cada una de ellas. También se expondrán 
las tendencias actuales y como no, la propuesta mencionada en el capítulo anterior.

\section{Aplicaciones de reseñas cinematográficas}

En la actualidad, existen diversas aplicaciones de reseñas cinematográficas que gozan de gran 
popularidad entre los usuarios. A continuación, se describen algunas de las más destacadas:

\subsection{IMDb}

Internet Movie Database (IMDb) es una de las aplicaciones de referencia en el mundo del cine. En ella, 
los usuarios pueden encontrar información sobre películas, programas de televisión, actores, 
directores, guionistas, etc. Además, permite a los usuarios puntuar y comentar las películas que han 
visto, y ver las reseñas y puntuaciones de otros usuarios.

Entre las características más destacadas de IMDb se encuentran su enorme base de datos, su sistema de 
puntuación basado en estrellas y su función de recomendaciones personalizadas.

\subsection{Rotten Tomatoes}

Rotten Tomatoes es otra de las aplicaciones más populares en el campo de las reseñas cinematográficas. 
En ella, los usuarios pueden ver la puntuación que la crítica y los usuarios han dado a una película, 
así como leer reseñas de expertos y de otros usuarios.

Entre las características más destacadas de Rotten Tomatoes se encuentran su sistema de puntuación 
basado en un porcentaje de opiniones positivas, su función de búsqueda avanzada y su sección de 
próximos estrenos.

\subsection{Letterboxd}

Letterboxd es una aplicación móvil de reseñas cinematográficas que destaca por su comunidad de 
usuarios y su enfoque en el cine independiente y de autor. En ella, los usuarios pueden puntuar y 
comentar las películas que han visto, crear listas personalizadas, seguir a otros usuarios y descubrir 
nuevas películas y críticas.

Entre las características más destacadas de Letterboxd se encuentran su diseño visual atractivo, su 
comunidad de usuarios comprometidos y su enfoque en el cine de autor.


\section{Comparación entre aplicaciones}

En esta sección se presentarán las principales similitudes y diferencias entre las aplicaciones 
expuestas en la sección anterior.

\subsection{Similitudes}

Todas estas aplicaciones permiten a los usuarios buscar películas y series por su título, género, año 
de estreno o elenco. A su vez, los usuarios pueden buscar información detallada sobre películas y 
series, como tráileres, sinopsis, reparto y calificaciones de los usuarios. También permiten a los 
usuarios crear y mantener listas de películas, serie y programas de televisión que han visto, que 
desean ver o que están viendo actualmente. Todas ellas tienen una sección de críticas y comentarios de 
los usuarios, donde los usuarios pueden exponer sus pensamientos sobre películas, series o programas 
de televisión.


\subsection{Diferencias}

Rotten Tomatoes se centra en la agregación de críticas de expertos y de la audiencia para proporcionar 
una calificación crítica global para cada película, serie o programa de TV, mientras que IMDb y 
Letterboxd se centran en las críticas y calificaciones de los usuarios individuales.

IMDb tiene una gran base de datos de información de la industria cinematográfica y televisiva, 
noticias, estrenos y detalles de producción, que las otras dos aplicaciones no tienen.

Letterboxd se centra en la comunidad y en proporcionar una plataforma para que los usuarios compartan 
y descubran películas, series y programas, mientras que IMDb y Rotten Tomatoes se centran más en 
proporcionar información y recomendaciones sobre películas y series.

También se diferencian en su diseño y funcionalidad, como la interfaz de usuario y su navegación

\subsection{Ventajas y Desventajas}

\subsubsection{IMDb}

\begin{itemize}
    \item \textbf{Ventajas}: Una amplia variedad de contenido, gran cantidad de reseñas y 
    calificaciones, información detallada de reparto y equipo de producción que ayuda a los usuarios 
    aprender mucho sobre las películas y series.
    \item \textbf{Desventajas}: Debido a la gran cantidad de información, la interfaz de usuario puede 
    resultar abrumadora, la calidad de las reseñas de usuarios puede ser variable y no siempre 
    confiable.
\end{itemize}

\subsubsection{Rotten Tomatoes}

\begin{itemize}
    \item \textbf{Ventajas}:Nos ofrece una puntuación ``\textit{fresca}'' o ``\textit{podrida}'' 
    basada  en las reseñas de los críticos, lo que puede ayudar a tomar decisiones más informadas por 
    parte de los usuarios. También tiene una comunidad activa que puede dejar sus reseñas y 
    calificaciones. De esta manera combina opiniones de críticos y usuarios, proporcionando una visión 
    más completa de una película o serie.
    \item \textbf{Desventajas}: Solo se limita a las reseñas de los profesionales y los usuarios 
    registrados, carece de información de reparto y equipo de producción. Además, algunos usuarios 
    piensas que la puntuación de Rotten Tomatoes se ha vuelto menos precisa con el tiempo.
\end{itemize}

\subsubsection{Letterboxd}

\begin{itemize}
    \item \textbf{Ventajas}: Posee una interfaz de usuario intuitiva, una comunidad bastante activa y 
    comprometida con el cine, también nos permite personalizar las lista y las recomendaciones.
    \item \textbf{Desventajas}:Al igual que Rotten Tomatoes no posee información detallada del reparto 
    o la producción, simplemente está limitado a las reseñas y puntuación por parte de los usuarios.
\end{itemize}

Cada aplicación posee sus ventajas y desventajas, y la elección de una sobre otra recae en gran medida 
de las necesidades y preferencias del usuario.

\subsection{API o Biblioteca}

Disponemos de herramientas que nos permiten acceder a información y datos de estas aplicaciones de 
reseñas. IMDb API, nos ofrece una inmensa base de datos de películas, calificaciones, reseñas y mucha 
más información. Permitiéndonos realizar búsquedas, obtener detalles de películas específicas, acceder 
a reseñas y a calificaciones de usuarios. Rotten Tomatoes API también nos permite acceder a 
información sobre las películas, como críticas de expertos o calificaciones de la audiencia. Por 
último, Letterboxd API también nos ofrece cierta información sobre sus películas, como reseñas o las 
listas de reproducción de los usuarios. Todas estas APIs proporcionan funcionalidades para integrar 
datos de reseñas u obtener otra información sobre la película deseada. Sin embargo, es importante 
tener en cuenta los términos de uso y las limitaciones de cada API antes de usarla.

\section{Tendencias actuales}

Referente a las tendencias actuales en las aplicaciones de reseñas, podemos encontrar una mayor 
integración de las redes sociales, ya que cada vez son más las aplicaciones que permiten compartir 
reseñas y valoraciones en redes sociales, fomentando la interacción entre usuarios. El uso de la 
inteligencia artificial y el \begin{otherlanguage}{english}``\textit{machine 
learning}''\end{otherlanguage} \cite{MachLear} están ayudando a las distintas aplicaciones a mejorar 
las recomendaciones personalizadas a los usuarios, basándose en su historial de visualización y 
valoraciones. También se están incluyendo nuevas formas de valoración, además de valorar mediante 
estrellas o puntuación numérica, algunas aplicaciones están incorporando nuevas formas de valoración, 
como emoticonos o valoraciones en forma de pulgar hacia arriba o hacia abajo, como un tipo de 
\begin{otherlanguage}{english}check \end{otherlanguage} verde o una equis roja. Además de las 
puntuaciones y reseñas, las aplicaciones están integrando más contenido referente a películas y 
series, como pueden ser los tráileres, las entrevistas al reparto o la dirección y más material 
relacionado. Cabe destacar el fomento de la diversidad y la inclusión debido a la demanda por parte de 
los usuarios, incluyendo así más diversidad en las recomendaciones y valoraciones, destacando también 
películas o series que aborden temas de inclusión y representatividad.

\subsection{Propuesta}

Tras el análisis de las aplicaciones de reseñas más usadas, ver sus pros, sus contras y las tendencias 
actuales, podemos construir una propuesta basada en las características más destacadas de las 
aplicaciones de reseñas cinematográficas:

\begin{itemize}
\item Base de datos amplia y actualizada
\item Sistema de puntuación y comentarios de usuarios
\item Sistemas de recomendación personalizados
\item Sección de próximos estrenos
\item Búsqueda avanzada y filtros de búsqueda
\item Comunidad de usuarios y funciones sociales
\item Enfoque en géneros específicos o en el cine de autor
\end{itemize}

De esta manera, para la solución propuesta en el segundo capítulo se deben tener en cuenta estas 
características a la hora de diseñar dicha solución, ya que permitirán a los usuarios tener una 
experiencia satisfactoria y completa al utilizar la aplicación.
