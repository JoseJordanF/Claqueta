\chapter{Estado del arte}


Es este capítulo se va a contemplar el mercado entre el que va a nadar nuestra solución. Un entorno donde se explorara como asegurar la calidad de una reseña y su veracidad. Qué mecanismos usa la competencia de nuestra solución para paliar estos problemas. De la misma manera tendremos una descripción detallada de estas aplicaciones o páginas web destinadas a las reseñas cinematográficas más populares en la actualidad y las características que las hacen destacar.


\section{Aplicaciones de reseñas cinematográficas}

En la actualidad, existen diversas aplicaciones de reseñas cinematográficas que gozan de gran 
popularidad entre los usuarios. A continuación, se describen algunas de las más destacadas:

\subsection{IMDb}

Internet Movie Database (IMDb) es una de las aplicaciones de referencia en el mundo del cine. En ella, 
los usuarios pueden encontrar información sobre películas, programas de televisión, actores, 
directores, guionistas, etc. Además, permite a los usuarios puntuar y comentar las películas que han 
visto, y ver las reseñas y puntuaciones de otros usuarios.

Entre las características más destacadas de IMDb se encuentran su enorme base de datos, su sistema de 
puntuación basado en estrellas y su función de recomendaciones personalizadas.

\subsection{Rotten Tomatoes}

Rotten Tomatoes es otra de las aplicaciones más populares en el campo de las reseñas cinematográficas. 
En ella, los usuarios pueden ver la puntuación que la crítica y los usuarios han dado a una película, 
así como leer reseñas de expertos y de otros usuarios.

Entre las características más destacadas de Rotten Tomatoes se encuentran su sistema de puntuación 
basado en un porcentaje de opiniones positivas, su función de búsqueda avanzada y su sección de 
próximos estrenos.

\subsection{Letterboxd}

Letterboxd es una aplicación móvil de reseñas cinematográficas que destaca por su comunidad de 
usuarios y su enfoque en el cine independiente y de autor. En ella, los usuarios pueden puntuar y 
comentar las películas que han visto, crear listas personalizadas, seguir a otros usuarios y descubrir 
nuevas películas y críticas.

Entre las características más destacadas de Letterboxd se encuentran su diseño visual atractivo, su 
comunidad de usuarios comprometidos y su enfoque en el cine de autor. \vspace{1cm}



Todas estas aplicaciones permiten a los usuarios buscar películas y series por su título, género, año 
de estreno o elenco. A su vez, los usuarios pueden buscar información detallada sobre películas y 
series, como tráileres, sinopsis, reparto y calificaciones de los usuarios. También permiten a los 
usuarios crear y mantener listas de películas, serie y programas de televisión que han visto, que 
desean ver o que están viendo actualmente. Todas ellas tienen una sección de críticas y comentarios de 
los usuarios, donde los usuarios pueden exponer sus pensamientos sobre películas, series o programas 
de televisión.


Pero centrándonos en el objetivo principal del proyecto, necesitamos saber como estas aplicaciones aseguran la fiabilidad de sus críticas y la nota de sus películas. Pero antes debemos saber como se escribe una crítica, este caso según la academia de cine de New York \footnote{\url{https://www.nyfa.edu/student-resources/9-tips-for-writing-a-film-review/}} existen varios consejos esenciales que abordarían la definición de calidad de una reseña, de los cuales por diseño nos quedaremos con:


\begin{itemize}
\item \textbf{Ver la película}, o al menos una vez. Puede parecer una necedad, pero es imposible captar todos los detalles y pensamientos con un solo visionado. Por ello es preferible que sean dos visionados, siendo buena idea tomar notas durante él visionado. Haciendo de esta manera más fácil el proceso de escritura, pudiendo recordar de manera sencilla sus pensamientos y reacciones consultando sus notas 
\item \textbf{Expresa tus opiniones y apoya tus críticas}, los críticos profesionales no se cortan un pelo a la hora de decir a sus lectores si una película les ha parecido buena, mala o indiferente. De hecho, los lectores confían en los críticos cuyos gustos reflejan los suyos. Por ello debe asegurarse de respaldar sus opiniones con datos concretos: una actuación decepcionante, una trama ridícula, una fotografía espectacular, etcétera. Pudiendo así los críticos profesionales expresar por qué y cómo han llegado a su reseña.
\item \textbf{Habla de la interpretación}, muchos cinéfilos ocasionalmente se animan a ver una película si en ella actúa su actor favorito. Por ello debería dedicar un poco de tiempo hablar sobre las interpretaciones. Comentar la forma en la que los actores manejaron el guion, la dinámica de conjunto y muchas otras cosas pueden ayudar a describir como lo hicieron los actores.
\item \textbf{Menciona a directores, directores de fotografía y efectos especiales}, explique a sus lectores los aciertos y los errores de los directores, directores de fotografía, diseñadores y CGI. Lo que ha funcionado, lo que le ha sorprendido, lo que no ha cumplido las expectativas, son cuestiones que se pueden tener en cuenta en el cuerpo de la crítica. También ayuda tener conocimientos de cine, como escritura de guiones, producción y mucho más. Pudiendo así contrastar tus palabras.
\item \textbf{¡Sin spoilers!}, da a tus lectores alguna idea de la trama de la película, pero ten cuidado con los spoilers. El objetivo de una buena crítica es despertar el interés de los lectores por ver la película. No te emociones demasiado y se lo arruines, o al menos avisa antes.
\item \textbf{Relee, reescribe y edita}, corrige tu trabajo. Tus opiniones no se tomarán en serio si escribes mal el nombre del director o no eres capaz de componer una frase gramaticalmente correcta. Tómate tu tiempo para revisar tu ortografía y editar tu trabajo para que tenga fluidez organizativa.
\end{itemize}


Veamos de esta manera, como las plataformas presentadas cumplen o no con estos criterios, y si no lo hacen con estos, como lo hacen:


\begin{itemize}
\item \textbf{IMDb:} En esta plataforma se nos ponen una serie de reglas \footnote{\url{https://acortar.link/MSYUlC}} que nos indican que debemos incluir en nuestras reseñas y que no. Haciendo uso de varios tips mencionados anteriormente, como ver la película y hablar solo del título indicado o similares para su comparación, no realizar spoilers a menos que antes avises, y argumentar tus palabras. Por otra parte, también nos obliga a incluir mínimo 600 palabras a nuestra crítica y nos indica que no debemos incluir. Como hablar en otro idioma que no sea inglés, hablar sobre la reseña de otra persona, cualquier tipo de blasfemia, calumnias y difamación. Y cualquier tipo de dato personal como números de teléfono. Con cada reseña, el usuario puntúa la película, esta puntuación se va actualizando cada cierto tiempo al día. Siendo la puntuación del título un promedio ponderado, teniendo notas del uno al diez siendo multiplicadas por su ponderación, que en este caso son el número de personas que han votado con esa nota, podemos ver una gráfica que nos indica el número de votos y su porcentaje en la sección de puntuación de cada película de esta plataforma \footnote{\url{https://www.imdb.com/title/tt0114709/ratings/?ref_=tt_ov_rt}}. Aunque cabe destacar que este sistema de puntuación se cambia a veces, ya que cuando se detecta un posible boicot a una película deben cambiar el sistema para que la nota de dicha película y sus reseñas sigan siendo confiables. El problema es que por más que he buscado no he encontrado información sobre ese sistema que usan en solo esas ocasiones.
\item \textbf{Rotten Tomatoes:} Nos advierte\footnote{\url{https://editorial.rottentomatoes.com/otg-article/community-code-of-conduct}} como IMDb que no debemos publicar contenido que pueda considerarse calumnia, difamación, apología al odio, acoso, amenazas, ... Por otro lado, esta plataforma no te obliga a escribir una reseña para votar. Su sistema de votación es una escala \textbf{\textit{Likert}} \cite{Elikert} en este caso se usan cinco puntos entre las estrellas para indicar como de acuerdo o desacuerdo que estas. Luego con todas las respuestas se calcula el porcentaje asociado a la película. Pero Rotten Tomatoes toma medidas para maximizar la fiabilidad de sus reseñas. Tanto que sus títulos tienen dos tipos de notas \footnote{\url{https://www.rottentomatoes.com/about}} una nota asignada por los críticos profesionales o personal verificado por ellos, y otra nota asignada por la audiencia. La nota de los profesionales se divide entre críticos profesionales y revistas o plataformas de video online. Mientras que la nota de la audiencia se divide entre personas verificadas y personas que no lo están, esto es algo relativamente nuevo. Ya que al igual que IMDb esta plataforma ha sufrido de ataques para acabar con la popularidad de algún título antes incluso de que se estrenara. Por ello, anularon que se pudieran añadir reseñas mientras no se hubiera estrenado la película y tomaron medidas para verificar ciertas reseñas, teniendo en cuenta si habían o no visto la película. Consiguiendo que estos usuarios entregaran su entrada de cine a través de unas cuantas plataformas como  \textbf{\textit{AMC Theatres}} \footnote{\url{https://www.amctheatres.com/}} donde pueden comprar sus entradas. Estas plataformas trabajan \footnote{\url{https://editorial.rottentomatoes.com/article/introducing-verified-audience-score/}} con Rotten Tomatoes para asegurar un poco más la fiabilidad de las reseñas de sus usuarios.
\item \textbf{Letterboxd:} como las otras plataformas, Letterboxd también posee unas reglas a la hora de escribir alguna reseña en algún título \footnote{\url{https://acortar.link/iknrGr}}. Nos claman a respetar a todos los usuarios, ya que formamos parte de una comunidad y no debemos atacar a nuestros iguales y menos en nuestras reseñas. No debemos usar este espacio para acosar, intimidar, e incitar al odio, la violencia o la intolerancia. Siempre que vayas a realizar algún spoilers márcalo en tu reseña, para no destrozar la experiencia de otro usuario. El cálculo de la puntuación en este caso es muy parecida a IMDb, disponemos de cinco estrellas y sus respectivas mitades. De esta manera podemos indicar notas desde media estrella equivalente a \textbf{\textit{0.5}} hasta cinco estrellas, aumentado de media estrella en media estrella. De esta manera el cálculo es un promedio ponderado al igual que IMDb pero esta vez sobre cinco en lugar de diez. Aunque recientemente han cambiado un poco el sistema de calificación, pero solo se usa cuando se detectan una gran cantidad de calificaciones inusuales, generadas por algún ataque hacia ese título. Aun así, según un artículo \cite{CRletterboxd} reciente de la página \textbf{\textit{Daily dot}} destinada a medios digitales cubriendo cultura y vida en internet, nos avisa de que simplemente es una leve desviación. Que puede incluso perjudicar más a las películas locales.
\end{itemize}

Cabe destacar que las tres plataformas constan de un equipo que revisa ciertas críticas para asegurar la calidad e integridad de estas, a través de sus respectivos códigos de conducta. De esta manera, las críticas que escribimos pasan por un procesamiento posterior a ser publicadas. Asegurando que algún usuario no se haya saltado las reglas. Sin embargo, nuestra infraestructura no nos permite esto ultimo.

\subsection{API o Biblioteca}

Disponemos de herramientas que nos permiten acceder a información, a datos de estas aplicaciones de 
reseñas y a funcionalidades. IMDb API, nos ofrece una inmensa base de datos de películas, 
calificaciones, reseñas y mucha más información. Permitiéndonos realizar búsquedas, obtener detalles de 
películas específicas, acceder a reseñas y a calificaciones de usuarios. Rotten Tomatoes API también 
nos permite acceder a información sobre las películas, como críticas de expertos o calificaciones de la 
audiencia. Por último, Letterboxd API también nos ofrece cierta información sobre sus películas, como 
reseñas o las listas de reproducción de los usuarios. Todas estas APIs proporcionan funcionalidades 
para integrar datos de reseñas u obtener otra información sobre la película deseada. Sin embargo, es 
importante tener en cuenta los términos de uso y las limitaciones de cada API antes de usarla.

\section{Tendencias actuales}

Referente a las tendencias actuales en las aplicaciones de reseñas, podemos encontrar una mayor integración de las redes sociales, ya que cada vez son más las aplicaciones que permiten compartir reseñas y valoraciones en redes sociales, fomentando la interacción entre usuarios. El uso de la inteligencia artificial\cite{ReviewsfAI} y el \begin{otherlanguage}{english}``\textit{machine learning}''\end{otherlanguage} \cite{MachLear} están ayudando a las distintas aplicaciones a mejorar\footnote{\url{https://acortar.link/qzhLqB}} las recomendaciones personalizadas a los usuarios, basándose en su historial de visualización y valoraciones. También se están incluyendo nuevas formas de valoración, además de valorar mediante estrellas o puntuación numérica, algunas aplicaciones están incorporando nuevas formas de valoración, como emoticonos o valoraciones en forma de pulgar hacia arriba o hacia abajo, como un tipo de \begin{otherlanguage}{english}check \end{otherlanguage} verde o una equis roja. Cabe destacar el fomento de la diversidad y la inclusión debido a la demanda por parte de los usuarios, incluyendo así más diversidad en las recomendaciones y valoraciones, destacando también películas o series que aborden temas de inclusión y representatividad.

El objetivo principal de nuestro proyecto es asegurar la calidad de las reseñas de los usuarios, algo muy importante que se mencionó en la introducción. La definición de esta calidad está formada por dos pilares, la fiabilidad de estas reseñas junto a los consejos que se han visto en este capítulo. Por ello nos centramos formas de asegurar que se sigan estos consejos y mecanismos que aseguren la fiabilidad de las reseñas.  




