\chapter{Planificación}

Este capítulo tiene como objetivo proporcionar una visión detallada de la planificación llevada a cabo para la realización de este Trabajo de Fin de Grado (TFG). Aquí se presentarán las diferentes secciones que abarcan el enfoque utilizado, el seguimiento del desarrollo del proyecto y las buenas prácticas que garanticen la calidad del proyecto y sirvan de indicador para seguir de cerca el desarrollo de este.

La planificación desempeña un papel fundamental en la consecución exitosa de cualquier proyecto académico, y el TFG no es una excepción. La adecuada organización de recursos, tiempos y tareas resulta esencial para alcanzar los objetivos propuestos y cumplir con los plazos establecidos.

En la primera sección, se describirá en detalle la mentalidad utilizada para abordar el desarrollo de este proyecto. Se explicarán las técnicas, herramientas y enfoques empleados, así como las razones detrás de su elección. Esta sección proporcionará una base sólida que sustenta todo el proceso de desarrollo del TFG.

La segunda sección abordará el seguimiento del desarrollo del proyecto. Se describirán las estrategias implementadas para monitorear el progreso, evaluar el cumplimiento de los objetivos y realizar ajustes en caso necesario. El seguimiento del desarrollo permitirá una supervisión constante, identificar posibles desviaciones y tomar medidas correctivas oportunas. También se abordará la temporización, donde gracias a los hitos clave del proyecto que reflejan la distribución del tiempo. Se permitirá una gestión eficiente del tiempo y una visualización clara de los avances y metas a alcanzar.

En resumen, este capítulo de planificación proporcionará una guía detallada sobre cómo se estructura y organiza este TFG desde el punto de vista metodológico, temporal y de seguimiento. La planificación adecuada y el control eficiente del desarrollo son elementos clave para garantizar el éxito y la calidad de este trabajo académico.

\section{Enfoque utilizado}

En esta sección, se presentará la mentalidad utilizada en el desarrollo de este TFG, centrada en el enfoque ágil. Se explicarán en detalle los conceptos, características y ventajas de este enfoque, así como la razón detrás de su elección para este proyecto. A continuación, se proporcionará una conclusión sobre por qué el enfoque ágil es adecuado y beneficioso para este TFG.

\subsection{Concepto y características}

El desarrollo ágil es un enfoque metodológico utilizado en proyectos de desarrollo de software que se caracteriza por su flexibilidad, adaptabilidad y enfoque iterativo e incremental. A diferencia de los enfoques tradicionales, que se basan en planes detallados y rigidez en los procesos, el desarrollo ágil se centra en la colaboración, la comunicación constante y la capacidad de respuesta a los cambios. Su principal objetivo es entregar un producto de calidad que cumpla con las necesidades y expectativas del cliente.

El desarrollo ágil se basa en los siguientes principios:

\begin{itemize}
\item \textbf{Orientación al cliente:} Se busca comprender las necesidades y expectativas del cliente y desarrollar soluciones que satisfagan sus requerimientos. La retroalimentación del cliente es fundamental para guiar el desarrollo y asegurarse de que el producto final cumpla con sus necesidades.
Por ello se han presentado los usuarios y se han visto reflejadas sus necesidades en las historias de usuario, descritas en el segundo capítulo junto al problema. Además de la reflexión del primer capítulo.

\item \textbf{Enfoque iterativo:} El proyecto se divide en iteraciones cortas y enfocadas, llamadas sprints, que tienen una duración definida (por ejemplo, 1 a 4 semanas). Cada sprint tiene objetivos claros y entrega un incremento funcional del producto. Al final de cada sprint, se revisa y se ajusta el plan en función de la retroalimentación y los resultados obtenidos.
A través de hitos, usando las herramientas ofrecidas por GitHub descritas más adelante en este capítulo, para definir dichos sprints, obteniendo en cada uno un producto mínimamente viable (PMV). Analizando y testeando si se obtienen los resultados esperados.

\item \textbf{Adaptabilidad y flexibilidad:} El desarrollo ágil reconoce que los requisitos y las prioridades pueden cambiar a lo largo del proyecto. En lugar de intentar prever y especificar todos los detalles desde el principio, se acepta que algunos requisitos pueden ser ambiguos o desconocidos al principio y se permite ajustar y adaptar el plan en función de las necesidades que surjan durante el desarrollo.
Esto se ve reflejado en los objetivos principales del proyecto, durante el desarrollo del proyecto algunos requisitos han ido cambiando según las necesidades que han surgido. Añadiendo o desechando ideas desde que empezó el proyecto.

\item \textbf{Entrega temprana y continua:} El enfoque ágil se enfoca en generar valor para el cliente de manera temprana y constante. Se priorizan las funcionalidades más importantes y se entregan en cada sprint, lo que permite obtener retroalimentación rápida y garantiza que el producto final cumpla con las expectativas del cliente.
Para ello se hace uso de \begin{otherlanguage}
{english}``\textit{\textbf{Pull Request}}''\end{otherlanguage} otra herramienta de GitHub descrita más abajo, por la que se tienen entregas continuas y su retroalimentación constante. Haciendo mucho más fácil y cómoda la resolución de cambios en él esas partes del proyecto.

\end{itemize}

El desarrollo ágil se basa en la premisa de que los requerimientos pueden cambiar, los problemas pueden surgir y las soluciones pueden evolucionar a lo largo del proyecto. En lugar de resistir a estos cambios, el enfoque ágil los abraza y busca manejarlos de manera efectiva a través de la colaboración, la iteración y la adaptabilidad.

\subsection{Ventajas}

Las ventajas, por lo general, que ofrece el desarrollo ágil sobre metodologías tradicionales:

\begin{itemize}
\item \textbf{Mayor satisfacción del cliente:} La entrega temprana y continua de incrementos funcionales permite obtener retroalimentación del cliente de forma constante, lo que garantiza que el producto final cumpla con sus expectativas y necesidades.
\item \textbf{Mayor visibilidad y control:} El enfoque ágil proporciona una mayor visibilidad del progreso del proyecto y permite un mayor control sobre el desarrollo, ya que se realizan seguimientos regulares y se realizan ajustes en función de la retroalimentación y los resultados obtenidos.
\item \textbf{Mejora en la calidad del producto:} La iteración constante y las pruebas frecuentes permiten identificar y corregir rápidamente los problemas, lo que conduce a un producto final de mayor calidad.
\end{itemize}

las ventajas bajo mi experiencia más importantes y destacables, serían la facilidad de adaptarse a los cambios a medida que avanzas en el proyecto y la comodidad de los pequeños entregables al usuario de forma continua y temprana, facilitando la organización de tareas. Trabajando de manera más simple sobre las correcciones dada la continua retroalimentación, aumentando la calidad del trabajo constantemente. Además, teniendo en cuenta que estás enfocado en satisfacer al usuario, tienes mayor dominio y transparencia del progreso del proyecto. Resumiendo, el enfoque ágil me ha permitido desarrollar de manera más efectiva mi TFG, y logrando una gran calidad.


\subsection{Conclusión}

Basándonos en las características y ventajas expuestas, se ha elegido el enfoque ágil como el enfoque principal para el desarrollo de este TFG. La naturaleza iterativa, adaptable y colaborativa del desarrollo ágil se alinea de manera efectiva con los objetivos y requisitos del proyecto.

El enfoque ágil permitirá una mayor flexibilidad para adaptarse a posibles cambios en los requerimientos, una entrega temprana de funcionalidades y una interacción continua con los usuarios finales. Además, facilitará la identificación temprana de problemas y la mejora constante de la calidad del producto.

En resumen, el enfoque ágil se considera una elección sólida para este TFG, ya que proporciona una estructura flexible y eficiente que permitirá una gestión efectiva del proyecto, una entrega de valor continua y una alta probabilidad de alcanzar los objetivos propuestos.


\section{Seguimiento del desarrollo}

A continuación se van a describir el conjunto de herramientas utilizadas durante el desarrollo de este proyecto.

\subsection{GitHub}

Siendo Git y GitHub las primeras herramientas fundamentales de este proyecto de software libre. Ya que permiten un seguimiento exhaustivo de los avances tanto en el código como en la documentación, los cuales están alojados en un repositorio público \footnote{\url{https://github.com/JoseJordanF/Claqueta}}. Además, facilitan la recuperación de versiones anteriores del software en caso de fallos o cambios significativos, lo cual resulta esencial en el contexto de un enfoque ágil que implica modificaciones constantes.

El repositorio de GitHub se convierte en la guía del proceso de desarrollo, proporcionando todas las herramientas necesarias para un seguimiento preciso del estado del trabajo en cualquier momento. A continuación, se detallan las principales funcionalidades utilizadas:

\begin{itemize}
\item \textbf{Issues} Se han creado issues \footnote{\url{https://github.com/JoseJordanF/Claqueta/issues}} en el repositorio de GitHub para realizar un seguimiento de las historias de usuario y las tareas asociadas a ellas. Estos issues describen el trabajo que debe realizarse y están orientados a satisfacer las necesidades de las historias de usuario. Cada vez que surja la necesidad de desarrollar algo o solucionar un problema, se documentará en un issue que indique los objetivos a lograr, y posteriormente se trabajará para cumplirlos.
\item \textbf{Commits} Cada commit representa un avance en el código o en la documentación. En el mensaje del commit se proporciona una breve descripción de los cambios realizados para avanzar en el issue al que se hace referencia. Esto permite conocer las decisiones tomadas para resolver una tarea específica y proporciona un historial completo de los cambios efectuados.
\item \textbf{Milestones} Los hitos se documentan en la sección de milestones, donde se resumen los objetivos que se espera alcanzar al tener un producto funcional. Estos hitos establecen fechas límite y enumeran los issues necesarios para completarlos en su totalidad. A medida que se vayan completando los issues, se podrá observar el progreso que falta por realizar.
\item \textbf{Pull Requests} Todo el código y la documentación que se encuentran en la rama principal del repositorio se considera probado y completamente funcional. El desarrollo se lleva a cabo en ramas separadas que están vinculadas a cada hito que se pretende alcanzar. Una vez que se completa una tarea de desarrollo, se crea un pull request hacia la rama principal. Si el pull request pasa satisfactoriamente los flujos de CI configurados, se considera que la tarea está completada y se incorpora una nueva versión funcional en la rama principal.
\end{itemize}

En resumen, GitHub proporciona un entorno centralizado donde se encuentran todas las herramientas necesarias para el desarrollo, incluyendo el código, la documentación y el seguimiento del progreso. Esto permite avanzar de manera eficiente hacia los objetivos establecidos y cumplir con las necesidades del usuario de forma efectiva.

\section{Hitos}

En esta mentalidad, se prioriza la flexibilidad y la adaptabilidad a medida que se avanza en el desarrollo del proyecto.

En lugar de una planificación exhaustiva y detallada desde el principio, se trabaja en iteraciones o sprints más cortos. Cada iteración tiene una duración definida y al final de cada una se entrega un incremento funcional del producto.

La temporización en el desarrollo ágil se ajusta según el progreso real del proyecto. Durante el transcurso de cada iteración, se evalúa el avance y se realizan ajustes en la planificación si es necesario. Esto permite adaptarse a cambios y prioridades emergentes de manera eficiente.

Por ello debemos tener claro que son los hitos, se pueden contemplar como objetivos claros y medibles que indican el avance del proyecto. La idea principal detrás de los hitos en el desarrollo ágil es tener puntos de control y evaluación frecuentes para asegurarse de que el proyecto está avanzando en la dirección correcta y cumpliendo con los objetivos establecidos. Además, los hitos proporcionan una oportunidad para realizar ajustes y adaptaciones en función de la retroalimentación y los resultados obtenidos en cada sprint. 

De esta manera, los hitos van a ser utilizados para marcar los logros clave que se deben alcanzar en diferentes etapas del proyecto. Estos hitos van a estar relacionados con la investigación, el diseño, la implementación o cualquier otra fase importante del TFG. Más específicamente para este proyecto vamos a trabajar con PMV (productos mínimamente viables), cada hito corresponderá a un PMV. Así cada hito que se defina se conformara por varios issues, a raíz de resolver dichos problemas iremos avanzando en hito. Completando uno tras otro, en un orden concreto como ir desde el más sencillo hasta el más complejo, obteniendo así un PMV, avanzando así en el proyecto. Un ejemplo de esto sería el hito de configuración del proyecto.

\subsection{M0. Configuración inicial del TFG - Estructura, objetivos y metodología usada}
El primer hito llevará a cabo la planificación inicial y su disposición para verificar la calidad del
proyecto.
El objetivo principal de este hito inicial es establecer la estructura del Trabajo de Fin de Grado (TFG) y preparar el proyecto. Después de completar este hito, se espera que los primeros puntos de la documentación estén redactados. Estos puntos incluirán la descripción del problema a abordar, el público objetivo afectado por dicho problema, la posible solución propuesta como producto final y el público al que va dirigido. Además, se abordará el estado actual del conocimiento en el área y el enfoque utilizado para desarrollar el producto. Asimismo, se configurará un repositorio que incluirá un verificador ortográfico para la documentación, los sistemas de CI y los problemas (issues), las historias de usuario y los hitos del proyecto.

\section{Elección de herramientas para el desarrollo}

Para el desarrollo de un proyecto de ingeniería del software, es indispensable el uso de herramientas que permitan llevar a cabo dicho proyecto, asegurando su calidad y buenas prácticas durante el uso de estas.

Por ello se van a describir a continuación todas las herramientas que se van a utilizar para el desarrollo del software. Describiendo los lenguajes de programación que se usaran, lenguajes de consulta y manipulación de datos para API, el modelo de datos a usar, a su vez también se mostraran herramientas que velen por el buen desarrollo en el repositorio y llevar a cabo buenas prácticas. El uso de todas estas herramientas será justificado, explicándose así para qué se va a utilizar dicha herramienta y porque se ha elegido.

\subsection{Lenguaje de programación}

Una de las principales herramientas para el desarrollo de software es el lenguaje de programación, un lenguaje que sea afín a las necesidades del proyecto. Siendo así un necesario un lenguaje para modelar los objetos descritos en este capítulo, en el primer hito. Pero debemos tener en cuenta que se pretende que el desarrollo del modelado de los objetos pueda ser reutilizable, de fácil acceso e intentando ahorrar recursos. Por ello, para la aplicación se pretende crear una API propia, como las que se mencionaron en el capítulo del estado del arte. Dichas API poseen una gran cantidad de datos que no necesitamos, por ello se desarrolla una API propia. ¿Pero qué tipo de API es la idónea para este proyecto?. Por lo que indagando por los tipos de API más usados, encontramos una gráfica \footnote{\url{https://acortar.link/mCfcHJ}}, en la página \textbf{''Postman''} reconocida por ser una web destinada para que los desarrolladores diseñen, construyan e iteren sus API. En dicha gráfica se ve como la REST es la arquitectura más utilizada. Esta arquitectura se compone de una serie de verbos http, los cuales son: 
\begin{itemize}
\item \textbf{GET:} se utiliza para consultar recursos sin causar efectos secundarios en el servidor. No crea ni modifica registros existentes
\item \textbf{POST:} se utilizan exclusivamente para crear nuevos recursos. Cada llamada con POST debe generar un recurso nuevo.
\item \textbf{PUT/PATCH:} se utilizan para modificar un recurso existente. PUT se diferencia de PATCH en que PUT reemplaza completamente el recurso, mientras que PATCH actualiza solo algunos elementos sin reemplazarlo por completo.
\item \textbf{DELETE:} es el verbo utilizado para eliminar registros, ya sea para eliminar un recurso individual o colecciones completas.
\end{itemize}

De esta manera podemos recuperar, insertar, modificar o eliminar información. Es importante destacar que REST está basado en múltiples \begin{otherlanguage}{english}\textit{\textbf{``endpoints''}}\end{otherlanguage}, estos no son más que URL o rutas específicas que se usan en la API REST para manipular los recursos que nos ofrece.

Por otra parte, también se pretende crear una aplicación que consuma esta API propia. Con lo mencionado en el capítulo dos podríamos crear una aplicación Android o una aplicación web, destinada a los dispositivos móviles. Según nos menciona \textbf{\textit{Cloud Levante}} en este artículo \footnote{\url{https://acortar.link/zynAAq}} lo más usado en móvil son las aplicaciones móviles. Por tanto, necesitamos un lenguaje de programación para aplicaciones móviles, más concretamente para Android, como mencionamos en el segundo capítulo. Tenemos varias opciones, consultando varios blogs\footnote{\url{https://blog.mgpanel.org/post/lenguajes-de-programacion-mas-usados-para-app-moviles}} todos ellos coinciden con qué Java es el primer lenguaje en el que piensas a la hora de crear una aplicación para Android. Sin embargo, no es Java el más utilizado, sino una variante de este, Kotlin. Debido a su sencillez y facilidad de uso, gracias a la simplicidad de su código será más simple crear tus funciones, incluyendo las corrutinas que facilitan la ejecución de tareas asíncronas. Destaca su extenso conjunto de librerías y bibliotecas, ya que puede usar las bibliotecas de java debido a la interoperabilidad de ambos lenguajes. Mencionar que \textit{Google} ha marcado a Kotlin como lenguaje recomendado para aplicaciones Android.

Algo muy importante es que Kotlin es un lenguaje de propósito general, lo cual nos permite crear diversos proyectos, entre los que se encuentra el \begin{otherlanguage}
{english}\textit{\textbf{Backend}}\end{otherlanguage} de una aplicación. Pudiendo crear nuestra API REST con Kotlin, gracias a diferentes \begin{otherlanguage}
{english}\textit{\textbf{Frameworks}}\end{otherlanguage} \footnote{\url{https://es.wikipedia.org/wiki/Framework}}. Varios de estos módulos de \begin{otherlanguage}
{english}\textit{software}\end{otherlanguage} nos permiten crear nuestra API REST. Entre ellos, el más utilizado era \begin{otherlanguage}
{english}\textbf{\textit{Spring Boot}}\end{otherlanguage}, que está siendo remplazado por otro módulo llamado \textbf{\textit{Ktor}} \footnote{\url{https://medium.com/@chaewonkong/ktor-the-next-generation-framework-that-might-replace-spring-boot-868e8d21fc0f}}. Este será el \begin{otherlanguage}
{english}\textit{Framework}\end{otherlanguage} más usado en Kotlin debido a su curva de aprendizaje, su alto rendimiento, ser ligero y modular, permitiendo incluir solo los componentes que el desarrollador necesita. Y algo relevante es un \begin{otherlanguage}
{english}\textit{framework}\end{otherlanguage} que se centra en Kotlin, ya que \begin{otherlanguage}
{english}\textit{spring boot}\end{otherlanguage} lo hace en Java y esto lo hace menos idiomático que Ktor.

Por lo tanto, al ser un lenguaje de propósito general, que nos permite tanto crear nuestra aplicación Android como un \begin{otherlanguage}
{english}\textit{backend}\end{otherlanguage}, sea sé nuestra API REST, y debido a su popularidad, facilidad de uso y la gran cantidad de recursos que nos ofrece. Kotlin es el lenguaje más indicado para el proyecto.


\subsection{Testing}

Un proyecto con un enfoque ágil está sujeto a pruebas constantemente, como ya sabemos de esta manera aseguramos la calidad del producto y nos cerciora de que todo funciona como debería. Consiguiendo así productos de calidad más robustos minimizando errores.

Como hemos mencionado, Kotlin goza de acceso a un extenso conjunto de librerías y \textit{frameworks}. En este conjunto existen varios \textit{frameworks} que nos permiten testear nuestro código.

Para test unitarios del código encontramos varios \textit{frameworks}, por un lado, tenemos \textit{Spek}\footnote{\url{https://github.com/spekframework/spek}}. Una herramienta escrita para Kotlin diseñado para facilitar la escritura y ejecución de pruebas en proyectos escritos en este lenguaje. Permite definir pruebas en un estilo legible similar al lenguaje natural, lo que facilita su comprensión tanto para desarrolladores como para no desarrolladores. Por ello, algunos desarrolladores lo relacionan con \begin{otherlanguage}
{english}\textit{Behavior-Driven Development}\end{otherlanguage} (BDD), desarrollo guiado por comportamiento. Aunque sus creadores ya han mencionado \footnote{\url{https://spekframework.github.io/spek/docs/latest/}} que creen que hay una falsa distinción en torno al desarrollo guiado por comportamiento (BDD) y desarrollo guiado por pruebas (TDD). Por lo que recomiendan que pensemos en Spek como un simple \textit{framework} de especificación.

También disponemos de la herramienta por defecto que incorpora cualquier tipo de proyecto Kotlin, \textit{JUnit5}. Este es la última versión del \textit{framework} de pruebas unitarias para Java. Posee una arquitectura modular que se compone de tres módulos principales: JUnit Platform, JUnit Jupiter y JUnit Vintage, el primero es el núcleo de la herramienta, el segundo introduce las anotaciones y permite configurar los test, y la última permite la compatibilidad con versiones anteriores de este \textit{framework}. Este es el más usado actualmente por los desarrolladores Android \footnote{\url{https://www.jetbrains.com/es-es/lp/devecosystem-2022/testing/}} como nos indica \textbf{\textit{Jetbrains}}, compañía que ha diseñado Kotlin.

Ambos tienen son buenas herramientas de pruebas. Además, permiten la integración con otras bibliotecas o \textit{framework} de pruebas. Pero ambas herramientas necesitan de otras bibliotecas imprescindibles en las pruebas, estas permiten simular objetos de una clase para trucar el resultado de ciertas funciones que queremos testear. Estos objetos se denominan \textbf{\textit{Mock}}, en Kotlin encontramos la librería nativa \textit{mockk}. Con el par de uno de los \textit{frameworks} mencionados y esta librería podríamos realizar los test unitarios que necesitemos. Pero principalmente si no en su totalidad usaremos Junit5 debido a la cantidad de información y ejemplos de uso, además de ser la usada por la mayoría de desarrolladores Android.

Para la parte de testeo de UI tenemos acceso a varias herramientas, pero nos limitaremos a usar el \textit{framework} \textbf{\textit{Espresso}}\footnote{\url{https://developer.android.com/training/testing/espresso?hl=es-419}}, una herramienta creada por Google y la más recomendada \footnote{\url{https://medium.com/mindful-engineering/ui-testing-with-espresso-in-android-10dfbc9f25da}}.

Por último, necesitamos una herramienta para testear las operaciones que creemos a través de los verbos http mencionados antes, para  recuperar, insertar, modificar o eliminar información gracias a nuestra API. Como hemos visto entre las herramientas más usadas para las pruebas\footnote{\url{https://www.jetbrains.com/es-es/lp/devecosystem-2022/testing/}} se encuentra \textit{\textbf{Postman}} que ya hemos mencionado que es una página para ayudar a los desarrolladores de API. Por ello es la herramienta que vamos a usar para realizar dichas pruebas, además es sencilla y cómoda de usar.







