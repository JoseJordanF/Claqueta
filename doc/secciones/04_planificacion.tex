\chapter{Planificación}

Este capítulo tiene como objetivo proporcionar una visión detallada de la planificación llevada a cabo para la realización de este Trabajo de Fin de Grado (TFG). Aquí se presentarán las diferentes secciones que abarcan la metodología utilizada, la temporización y el seguimiento del desarrollo del proyecto.

La planificación desempeña un papel fundamental en la consecución exitosa de cualquier proyecto académico, y el TFG no es una excepción. La adecuada organización de recursos, tiempos y tareas resulta esencial para alcanzar los objetivos propuestos y cumplir con los plazos establecidos.

En la primera sección, se describirá en detalle la metodología utilizada para abordar el desarrollo de este proyecto. Se explicarán las técnicas, herramientas y enfoques empleados, así como las razones detrás de su elección. Esta sección proporcionará una base sólida que sustenta todo el proceso de desarrollo del TFG.

La segunda sección abordará el seguimiento del desarrollo del proyecto. Se describirán las estrategias implementadas para monitorear el progreso, evaluar el cumplimiento de los objetivos y realizar ajustes en caso necesario. El seguimiento del desarrollo permitirá una supervisión constante, identificar posibles desviaciones y tomar medidas correctivas oportunas. También se abordará la temporización, donde gracias a los hitos clave del proyecto que reflejan la distribución del tiempo. Se permitirá una gestión eficiente del tiempo y una visualización clara de los avances y metas a alcanzar.

En resumen, este capítulo de planificación proporcionará una guía detallada sobre cómo se estructura y organiza este TFG desde el punto de vista metodológico, temporal y de seguimiento. La planificación adecuada y el control eficiente del desarrollo son elementos clave para garantizar el éxito y la calidad de este trabajo académico.

\section{Metodología utilizada}

En esta sección, se presentará la metodología utilizada en el desarrollo de este TFG, centrada en el enfoque ágil. Se explicarán en detalle los conceptos, características y ventajas de esta metodología, así como la razón detrás de su elección para este proyecto. A continuación, se proporcionará una conclusión sobre por qué el enfoque ágil es adecuado y beneficioso para este TFG.

\subsection{Concepto y Características}

El desarrollo ágil es un enfoque metodológico utilizado en proyectos de desarrollo de software que se caracteriza por su flexibilidad, adaptabilidad y enfoque iterativo e incremental. A diferencia de los enfoques tradicionales, que se basan en planes detallados y rigidez en los procesos, el desarrollo ágil se centra en la colaboración, la comunicación constante y la capacidad de respuesta a los cambios. Su principal objetivo es entregar un producto de calidad que cumpla con las necesidades y expectativas del cliente.

El desarrollo ágil se basa en los siguientes principios:

\begin{itemize}
\item \textbf{Colaboración estrecha:} Fomenta la interacción constante entre los miembros del equipo de desarrollo, los stakeholders y los usuarios finales. Se promueve la comunicación abierta, el intercambio de ideas y la resolución conjunta de problemas.
\item \textbf{Enfoque iterativo:} El proyecto se divide en iteraciones cortas y enfocadas, llamadas sprints, que tienen una duración definida (por ejemplo, 1 a 4 semanas). Cada sprint tiene objetivos claros y entrega un incremento funcional del producto. Al final de cada sprint, se revisa y se ajusta el plan en función de la retroalimentación y los resultados obtenidos.
\item \textbf{Adaptabilidad y flexibilidad:} El desarrollo ágil reconoce que los requisitos y las prioridades pueden cambiar a lo largo del proyecto. En lugar de intentar prever y especificar todos los detalles desde el principio, se acepta que algunos requisitos pueden ser ambiguos o desconocidos al principio y se permite ajustar y adaptar el plan en función de las necesidades que surjan durante el desarrollo.
\item \textbf{Entrega temprana y continua:} El enfoque ágil se enfoca en generar valor para el cliente de manera temprana y constante. Se priorizan las funcionalidades más importantes y se entregan en cada sprint, lo que permite obtener retroalimentación rápida y garantiza que el producto final cumpla con las expectativas del cliente.
\item \textbf{Orientación al cliente:} Se busca comprender las necesidades y expectativas del cliente y desarrollar soluciones que satisfagan sus requerimientos. La retroalimentación del cliente es fundamental para guiar el desarrollo y asegurarse de que el producto final cumpla con sus necesidades.
\end{itemize}

El desarrollo ágil se basa en la premisa de que los requerimientos pueden cambiar, los problemas pueden surgir y las soluciones pueden evolucionar a lo largo del proyecto. En lugar de resistir a estos cambios, el enfoque ágil los abraza y busca manejarlos de manera efectiva a través de la colaboración, la iteración y la adaptabilidad.

\subsection{Ventajas}

Las ventajas que ofrece el desarrollo ágil sobre metodologías tradicionales:

\begin{itemize}
\item \textbf{Mayor satisfacción del cliente:} La entrega temprana y continua de incrementos funcionales permite obtener retroalimentación del cliente de forma constante, lo que garantiza que el producto final cumpla con sus expectativas y necesidades.
\item \textbf{Mayor visibilidad y control:} El enfoque ágil proporciona una mayor visibilidad del progreso del proyecto y permite un mayor control sobre el desarrollo, ya que se realizan seguimientos regulares y se realizan ajustes en función de la retroalimentación y los resultados obtenidos.
\item \textbf{Mejora en la calidad del producto:} La iteración constante y las pruebas frecuentes permiten identificar y corregir rápidamente los problemas, lo que conduce a un producto final de mayor calidad.
\end{itemize}

\subsection{Conclusión}

Basándonos en las características y ventajas expuestas, se ha elegido el enfoque ágil como la metodología principal para el desarrollo de este TFG. La naturaleza iterativa, adaptable y colaborativa del desarrollo ágil se alinea de manera efectiva con los objetivos y requisitos del proyecto.

El enfoque ágil permitirá una mayor flexibilidad para adaptarse a posibles cambios en los requerimientos, una entrega temprana de funcionalidades y una interacción continua con los usuarios finales y los \begin{otherlanguage}
{english}``\textit{stakeholders}''\end{otherlanguage}(partes interesadas). Además, facilitará la identificación temprana de problemas y la mejora constante de la calidad del producto.

En resumen, el enfoque ágil se considera una elección sólida para este TFG, ya que proporciona una estructura flexible y eficiente que permitirá una gestión efectiva del proyecto, una entrega de valor continua y una alta probabilidad de alcanzar los objetivos propuestos.


\section{Seguimiento del desarrollo}

A continuación se van a describir el conjunto de herramientas utilizadas durante el desarrollo de este proyecto.

\subsection{GitHub}

Siendo Git y GitHub las primeras herramientas fundamentales de este proyecto de software libre. Ya que permiten un seguimiento exhaustivo de los avances tanto en el código como en la documentación, los cuales están alojados en un repositorio público \footnote{\url{https://github.com/JoseJordanF/Claqueta}}. Además, facilitan la recuperación de versiones anteriores del software en caso de fallos o cambios significativos, lo cual resulta esencial en el contexto de una metodología ágil que implica modificaciones constantes.

El repositorio de GitHub se convierte en la guía del proceso de desarrollo, proporcionando todas las herramientas necesarias para un seguimiento preciso del estado del trabajo en cualquier momento. A continuación, se detallan las principales funcionalidades utilizadas:

\subsubsection{Issues}

Se han creado issues en el repositorio de GitHub para realizar un seguimiento de las historias de usuario y las tareas asociadas a ellas. Estos issues describen el trabajo que debe realizarse y están orientados a satisfacer las necesidades de las historias de usuario. Cada vez que surja la necesidad de desarrollar algo o solucionar un problema, se documentará en un issue que indique los objetivos a lograr, y posteriormente se trabajará para cumplirlos.

\subsubsection{Milestones}

Los hitos se documentan en la sección de milestones, donde se resumen los objetivos que se espera alcanzar al tener un producto funcional. Estos hitos establecen fechas límite y enumeran los issues necesarios para completarlos en su totalidad. A medida que se vayan completando los issues, se podrá observar el progreso que falta por realizar.

\subsubsection{Commits}

Cada commit representa un avance en el código o en la documentación. En el mensaje del commit se proporciona una breve descripción de los cambios realizados para avanzar en el issue al que se hace referencia. Esto permite conocer las decisiones tomadas para resolver una tarea específica y proporciona un historial completo de los cambios efectuados.

\subsubsection{Pull Requests}

Todo el código y la documentación que se encuentran en la rama principal del repositorio se considera probado y completamente funcional. El desarrollo se lleva a cabo en ramas separadas que están vinculadas a cada hito que se pretende alcanzar. Una vez que se completa una tarea de desarrollo, se crea un pull request hacia la rama principal. Si el pull request pasa satisfactoriamente los flujos de CI configurados, se considera que la tarea está completada y se incorpora una nueva versión funcional en la rama principal.


En resumen, GitHub proporciona un entorno centralizado donde se encuentran todas las herramientas necesarias para el desarrollo, incluyendo el código, la documentación y el seguimiento del progreso. Esto permite avanzar de manera eficiente hacia los objetivos establecidos y cumplir con las necesidades del usuario de forma efectiva.

\section{Temporización}

En esta metodología, se prioriza la flexibilidad y la adaptabilidad a medida que se avanza en el desarrollo del proyecto.

En lugar de una planificación exhaustiva y detallada desde el principio, se trabaja en iteraciones o sprints más cortos. Cada iteración tiene una duración definida y al final de cada una se entrega un incremento funcional del producto.

La temporización en el desarrollo ágil se ajusta según el progreso real del proyecto. Durante el transcurso de cada iteración, se evalúa el avance y se realizan ajustes en la planificación si es necesario. Esto permite adaptarse a cambios y prioridades emergentes de manera eficiente.

En esta sección, se proporcionarán los detalles específicos de cada hito, incluyendo las fechas de inicio y fin correspondientes. Esto permitirá tener una visión clara del cronograma del proyecto y facilitará el seguimiento del progreso a lo largo del tiempo.

Por ello debemos tener claro que son los hitos, se pueden contemplar como objetivos claros y medibles que indican el avance del proyecto. La idea principal detrás de los hitos en el desarrollo ágil es tener puntos de control y evaluación frecuentes para asegurarse de que el proyecto está avanzando en la dirección correcta y cumpliendo con los objetivos establecidos. Además, los hitos proporcionan una oportunidad para realizar ajustes y adaptaciones en función de la retroalimentación y los resultados obtenidos en cada sprint. 

De esta manera, los hitos van a ser utilizados para marcar los logros clave que se deben alcanzar en diferentes etapas del proyecto. Estos hitos van a estar relacionados con la investigación, el diseño, la implementación o cualquier otra fase importante del TFG.

\subsection{M0. Configuración Inicial del TFG - Estructura, objetivos y metodología usada}

El objetivo principal de este hito inicial es establecer la estructura del Trabajo de Fin de Grado (TFG) y preparar el proyecto. Después de completar este hito, se espera que los primeros puntos de la documentación estén redactados. Estos puntos incluirán la descripción del problema a abordar, el público objetivo afectado por dicho problema, la posible solución propuesta como producto final y el público al que va dirigido. Además, se abordará el estado actual del conocimiento en el área y la metodología utilizada para desarrollar el producto. Asimismo, se configurará un repositorio que incluirá un verificador ortográfico para la documentación, los problemas (issues), las historias de usuario y los hitos del proyecto.

\subsection{M1. Definición de Objetos - Abstracción del proyecto}

En este hito, se abstraerán los conceptos clave y se definirán los objetos de la aplicación. El objetivo es tener una estructura clara de los datos a utilizar. La abstracción nos permite identificar las características esenciales, eliminando detalles innecesarios. La definición de objetos nos ayudará a comprender sus relaciones, atributos y operaciones. Esto establecerá una base sólida para el desarrollo coherente de la aplicación.

Para ello se han buscado varias formas de describir los objetos de la aplicación, como los diagramas UML, herramientas visuales utilizadas en ingeniería de software para representar y modelar sistemas y procesos. Proporcionan una representación gráfica de diferentes aspectos de un sistema, como su estructura, comportamiento, interacciones y relaciones entre componentes. Los diagramas UML se utilizan para comunicar y documentar el diseño de software de manera estándar y comprensible, facilitando la comprensión y colaboración entre los miembros del equipo de desarrollo.

Sin embargo, las desventajas de utilizar diagramas UML en un proyecto basado en desarrollo ágil puede no ser recomendable debido a la sobrecarga de documentación, la rigidez y falta de adaptabilidad, la comunicación ineficiente y el enfoque en la ejecución y la iteración. Siendo muy problemático que tienden a requerir que todo el diseño esté definido antes de comenzar la implementación. Esto puede limitar la capacidad de adaptarse a los cambios y de iterar rápidamente en el desarrollo.

En su lugar, se prioriza el desarrollo real, la retroalimentación continua y prácticas como el desarrollo impulsado por pruebas (TDD) y la colaboración cercana. Esto permite una mayor agilidad y adaptabilidad en el proceso de desarrollo.

Por ello, buscando otro método para definir los objetos de la aplicación, se encuentra una técnica de desarrollo de software llamada, \begin{otherlanguage}
{english}``\textit{\textbf{Domain Driven Design}}''\end{otherlanguage}(DDD) es un enfoque de diseño de software que se centra en comprender y modelar el dominio del problema de una aplicación. Busca desarrollar un diseño que refleje con precisión las reglas y conceptos del dominio, lo que resulta en un sistema más mantenible. Teniendo en cuenta que el DDD y el desarrollo ágil son compatibles, ya que su aplicación permite construir aplicaciones que se ajusten mejor a las necesidades del cliente y evolucionen de manera flexible a medida que se adquiere un mayor entendimiento del dominio. DDD proporciona la base conceptual y de diseño sólida para desarrollar modelos de dominio claros y significativos, mientras que el enfoque ágil permite una implementación iterativa, rápida y adaptativa.

De esta manera se recurre al uso del \textit{\textbf{Modelo de Dominio}} representación conceptual de las entidades, los conceptos, las reglas y las interacciones dentro del dominio del problema. Es una abstracción del mundo real que captura las principales entidades y sus relaciones. Esto ofrece una gran ventaja del DDD, ya que esto ayuda a alinear el entendimiento y facilita la comunicación efectiva sobre el problema y su solución. Siendo los modelos de dominio una parte central y fundamental del DDD, representando el conocimiento y la comprensión profunda del problema que se está abordando. Teniendo esto en cuenta se crea un modelo de dominio que se puede visualizar en este enlace \footnote{\url{https://acortar.link/iohTNI}}.

\subsection{M2. Lógica de Negocio - Operaciones sobre los datos}

La lógica de negocio, también conocida como reglas de negocio, se refiere a las operaciones y procesos fundamentales que definen cómo funciona la aplicación. Determinando como se procesan los datos, se realizan cálculos, se toman decisiones y se llevan a cabo las operaciones clave para lograr los objetivos de la aplicación. Siempre dependientes de las necesidades y los propósitos de la aplicación. Como estas vienen definidos por las \textbf{Historias de Usuario}, vamos a recurrir a ellas para determinar las operaciones que se llevaran a cabo con los datos de la aplicación. 

Por tanto, siguiendo las historias del aficionado al cine, el cinéfilo y el crítico, podemos definir estas operaciones:

\begin{itemize}
\item \textbf{Publicación y gestión de reseñas:} La lógica de negocio permitiría a los usuarios publicar reseñas de películas, incluyendo la calificación y por supuesto el contenido de la reseña en sí, como la opinión del usuario. También podría incluir la capacidad de editar o eliminar reseñas propias, así como la opción de comentar y responder a las reseñas de otros usuarios.
\item \textbf{Búsqueda y filtrado de películas:} En este caso se permitiría a los usuarios buscar películas por título, director, año de estreno, e incluso duración. Esto implicaría implementar algoritmos de búsqueda y criterios de filtrado para presentar los resultados más relevantes. 
\item \textbf{Recomendaciones personalizadas:} Utilizar técnicas de recomendación para ofrecer a los usuarios sugerencias personalizadas de películas basadas en su historial de reseñas o sus preferencias. En un futuro se podrían llegar a usar algoritmos de aprendizaje automático (algoritmos desarrollados para aprender de los datos y tomar decisiones sin ser programados explícitamente), que analicen los patrones de comportamiento del usuario y sugieran películas afines a sus gustos.
\item \textbf{Asegurar la fiabilidad:} Asegurar que la interacción social entre los usuarios, a través de las reseñas de estos, sea lo más fiable posible. Mediante un algoritmo que penalice o recompense individualmente a los usuarios por sus interacciones. Evaluando así la fiabilidad de dichas interacciones, siempre dependiendo del algoritmo que se use.
\end{itemize}

\subsection{M3. Lógica de Negocio - Operaciones sobre los datos}