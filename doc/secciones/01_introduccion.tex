

\chapter{Introducción}

Gran parte del entretenimiento se deriva del propio contenido audiovisual, ya sean series, películas,
documentales, recitales, conciertos, etc. Un grupo de personas simplemente se conforma con visualizar 
el contenido, pero existe otro grupo que no satisface su necesidad de entretenimiento solo visionando 
estos productos. Por ello necesitan algo más, ese algo es compartir su experiencia, su opinión acerca 
del producto consumido, con sus allegados y amigos. Pero como es posible que lo hagan con otros 
usuarios que sientan la misma necesidad de debatir o especular sobre la obra audiovisual que han visto.

A través de reseñas. \textit{''Una reseña cinematográfica es una reacción instantánea de un espectador 
a una película. Es como el chisme en una cena, cuando te vuelves hacia la persona de al lado y le 
preguntas ''¿qué pensaste?'' Una reseña de cine es una respuesta a una experiencia compartida de la 
película, no es una síntesis de los elementos que la componen''}. Definición de reseña por 
\textit{''Roger Ebert''} un reconocido escritor, y critico de cine de Estados Unidos \cite{REwebsite}. 

Esa necesidad de comentar la obra que acababas de visualizar, comenzó a finales del siglo XIX y a 
principios del siglo XX, ya que se popularizó el cine como forma de entretenimiento. Creando así la 
necesidad de comentar, analizar y discutir estas películas. Gracias a la concepción del cine como un 
arte y no solo como un entretenimiento a principio de los años 20 por los críticos franceses, entre los 
que se encontraba \textit{''Louis Delluc''}, periodista, director y critico francés \cite{LouisD}. 
Surgieron entre los años 20 y 30 críticos en otros países como Reino Unido y Estados Unidos como el 
ganador del premio Pulitzer en 1958 por su novela autobiográfica \textit{''James Agee''} \cite{JamesA}. 

En el titulado \begin{otherlanguage}{english}``\textit{A very short history of film 
criticism}''\end{otherlanguage}\cite{3BrothersArticle}. \footnote{El título del artículo y futuros 
términos en inglés serán marcados para preservar la calidad del texto y poder realizar comprobaciones 
ortográficas en distintos idiomas}

Encontramos como el autor cuenta como ha evolucionado la crítica cinematográfica desde la época del 
famoso crítico \textit{''Roger Ebert''} hasta la era de las redes sociales. El autor trata de explicar 
que la crítica cinematográfica ha seguido siendo importante y evoluciona a lo largo de los años. Remarca 
que la crítica cinematográfica era más elitista y se enfocaba en películas de arte y ensayo, un 
tipo de cine que no ofrece una narración dramática ni una representación histórica del mundo, sino una 
reflexión sobre el propio medio. Diferenciándose de la narrativa clásica en varios puntos, siendo 
narrativa de arte y ensayo frente a narrativa clásica. Casualidad frente a  causalidad, un final 
abierto frente a un final cerrado, tiempo no lineal frente a tiempo literal, conflicto interno frente a 
conflicto externo, un protagonista pasivo frente a un protagonista activo, una realidad ambigua frente 
a una realidad coherente y el centro de la historia es el personaje y no la acción mientras que por la 
narrativa clásica se centra en la acción y no en el personaje. Con el tiempo la crítica se extendió y 
se añadieron a su abanico las películas comerciales. La influencia de \textit{''Roger Ebert''} ayudo a 
popularizar la crítica cinematográfica en Estados Unidos gracias a sus escritos y a sus programas de 
televisión. 

A lo largo del tiempo, estos críticos han ayudado en gran medida a la industria cinematográfica 
haciéndoles un gran favor, ya que no solo han servido de guía para los espectadores, llevándolos hacia 
las películas más destacadas, gracias a su perspectiva crítica sobre estas. Si no que también han 
desempeñado un papel importante en el desarrollo de la teoría del cine y la crítica literaria. 
Influyendo así en la forma en que se piensa y se habla sobre el cine en la cultura general. Se podría 
decir que el desarrollo del cine fue el que permitió que apareciera el mundo de las críticas 
cinematográficas, siendo una parte importante de la industria e importante para construir nuestra parte 
crítica dirigida a estas obras cinematográficas. 

Las herramientas que se usan para realizar estas reseñas también se han desarrollado a lo largo del 
tiempo, desde las herramientas más básicas de escritura, papel y pluma, pasando por la máquina de 
escribir, uso de grabadoras y cámaras, la llegada del ordenador de sobremesa y llegando hasta el 
ordenador portátil. Estas herramientas han permitido que las opiniones de los críticos lleguen cada vez 
a más gente en el menos tiempo posible. El mayor salto fue la aparición de Internet, de manera que 
podían compartir su opinión llegando a casi todas las partes del mundo, incluso gracias a nuevas 
tecnologías que han ido apareciendo como el \begin{otherlanguage}
{english}``\textit{streamming}''\end{otherlanguage} o las redes sociales comentadas en el artículo 
mencionado con anterioridad, que ofrecen realizar criticas en tiempo real, siendo una herramienta con 
mucho potencial y al alcance de la mano de cualquier usuario.

De este modo existen varios espacios famosos como IMDb\cite{IMDbWeb}, web por excelencia conocida por 
ser una de las pioneras en Internet. Contiene una gran cantidad de información sobre todo lo 
relacionado con el cine y las series. Una masiva información de calidad al alcance de cualquier 
usuario, desde si buscas a un actor en particular, cierto director o simplemente quieres comentar lo 
mucho que te ha entusiasmado una película que acabas de ver. También posee un sistema de ranking 
mediante el cual los usuarios pueden puntuar cada reseña, estableciendo así cierta reputación a 
usuarios que tengan una puntuación decente gracias a sus opiniones fundadas sobre la obra 
cinematográfica. Otros espacios similares serían Rotten Tomatoes\cite{RottenTWeb} o una web española 
como SensaCine\cite{SSweb} en las que podemos realizar todas las actividades mencionadas antes.

Como se mencionó antes, la llegada de Internet y las redes sociales ha permitido que la crítica 
cinematográfica sea más accesible y democrática. Pero también ha traído una ingente cantidad de 
información, provocando una sobrecarga de información y una perdida de profundidad crítica, debido a la 
fugacidad y la celeridad de las redes sociales, han conducido a la crítica cinematográfica a que sea 
más superficial y menos reflexiva. Aun así, la crítica cinematográfica sigue siendo importante en la 
época de las redes sociales, ya que puede ayudar a sus consumidores a comprender mejor el mundo en el 
que viven, descubrir películas y directores interesantes. Por lo que el artículo mencionado 
anteriormente concluye afirmando que la critica, a pesar de haber cambiado a lo largo de los años, 
continúa siendo una forma valiosa de arte y análisis cultural, y anima a sus lectores a buscar y 
compartir reseñas sobre obras cinematográficas que visualicen.

Por ello se plantea un problema, los usuarios necesitan complacer su necesidad de compartir su opinión 
sobre toda obra cinematográfica que ven, sin impedimentos en cualquier lugar y en cualquier momento. 
Existen otros usuarios cuyas necesidades se encuentran en discernir si ven o no una película, esta 
decisión recae sobre esas opiniones de otros usuarios que ya han visto el contenido. Así, el usuario 
que necesita una recomendación para finalmente visualizar o no algún contenido audiovisual recurre a 
estas críticas, ya les agraden o no, les ayudan a tomar una decisión. Por lo que la reseña influye mucho 
en la deliberación de los usuarios dubitativos de ver cierto contenido.

\section{Motivación}

Desde muy pequeño siempre me ha gustado ver las cintas VHS que había en mi casa, desde que me las ponía 
mi familia, hasta que yo pude poner una y otra vez esas cintas, llegando a repetir la misma película 
infinidad de veces sin cansancio alguno. De alguna manera el cine es un mundo aparte en el que no hay 
nada más que la película que estás viendo en ese momento, no hay polémicas, problemas, agobios, estrés. 
Te desinhibes al completo y disfrutas de la película con los que te rodean, riendo, llorando, cantando 
o sintiendo, por esos personajes que hacemos nuestros cuando vemos alguna obra cinematográfica. 

Y de alguna manera el poder expresar lo que te ha parecido la obra que has visto y compartirlo con 
aquellas personas que se crucen con tu reseña y tengan la posibilidad de interactuar contigo y discutir 
sobre lo que ambos habéis visto; me hace volver a esos tiempos en los que estás con amigos o familia 
comentando tranquilamente en casa, una película en la televisión. Al final solo estás viendo ficción, 
comedia, tristeza, alegría, terror, vida, muerte. Siendo el objetivo de la obra, transmitir ese 
sentimiento y hacer que te evadas de lo demás. También es una herramienta muy fuerte en países con 
algún régimen opresor, siendo una manera de expresión, liberación, comunicación y protesta frente al 
gobierno que los oprime. Otro motivo es la integración cultural, existen idiomas o culturas que se 
pierden a lo largo del tiempo, si es posible resguardar dicho lenguaje y costumbres de aquellas 
culturas, aunque sea ficción, merecerá la pena para que persista en el tiempo y pueda ser visto por 
otros que no vivieron en esos tiempos. Recordando así tanto lo malo como lo bueno de la historia, 
pudiendo aprender de ello.

Y las reseñas sobre esas obras tendrían el poder de transmitir al resto del mundo el sentimiento 
producido por ese cine y conseguir que el público contribuya a la causa, presionando a la parte 
opresora o simplemente divulgando para que más gente disfrute de esas obras audiovisuales. Por tanto, 
el poder de la escritura, de la reseña en este caso, no es una minucia.


Cita de \textit{''Neil Gaiman''}\cite{NeilG} famoso escritor británico de ficción, novelas gráficas,
películas, teatro de audio...
\begin{verbatim}
    ``Solo escribiendo puedo hacer 
    que el mundo sea lo que debería ser''.
\end{verbatim}

Así, el problema en el que sé profundiza en el siguiente capítulo recae en la calidad de las reseñas, 
en su veracidad y en su influencia. Ya que con el auge de las plataformas digitales y las redes 
sociales, cualquier persona puede expresar su opinión sobre alguna serie o película sin necesariamente 
tener conocimientos de cine. Esto ha provocado una proliferación de reseñas de diversa calidad y 
credibilidad, lo que dificulta a los usuarios a diferenciar entre opiniones fundadas y expresiones 
subjetivas. Además, la influencia de las reseñas en la elección de series o películas ha aumentado 
significativamente, lo que puede llevar a una sobrevaloración o infravaloración de cierto contenido. 
Afectando esto la percepción general y a la industria cinematográfica.

Por ello se debe abordar este problema y buscar mecanismos que permitan medir la calidad o fiabilidad 
de las reseñas \cite{RRbyProfeMC}, para ayudar a los espectadores a tomar decisiones informadas \cite{RSforMovieRRating} y enriquecer su 
experiencia cinematográfica.

De esta manera, toda persona que disfrute con el cine tendrá la oportunidad también de compartir su 
experiencia tras la visualización de cualquier contenido, haciendo uso de una opinión fundamentada y 
fiable, instigando a otras personas a ver la obra o incluso a comentar lo que les ha parecido también a 
ellos. Por ello, gracias a la época tecnológica en la que vivimos, es posible encontrar una solución a 
esa necesidad de cualquier usuario de manifestar su postura frente a las obras audiovisuales que 
consuma, además de ayudar a la toma de decisiones a la hora de visionar contenido. Por lo que el motivo 
de este TFG es proporcionar una solución informática frente a esta necesidad de los usuarios, que sea 
útil, de manera que anime al usuario a usarla en su día a día para satisfacer su necesidad de compartir 
su experiencia, comentarlo con el resto de usuarios y hacer uso de ella cuando quiera una opinión de 
calidad para consumir o no cierto contenido. Siendo una solución sencilla y accesible para que todo 
usuario pueda utilizarla y acceder a ella de la forma más simple posible. Más debe ser una solución 
rentable y sostenible a largo plazo, y dispuesta a cambios suscitados por los usuarios que consumen 
dicha solución, para hacerla cada vez más robusta y apegada a las necesidades constantes de los 
usuarios. Pudiendo así ampliar la funcionalidad de la solución una vez resueltas las necesidades 
básicas. 

Este trabajo está dividido en tres partes diferenciadas:
\begin{itemize}
    \item En los \textbf{capítulos del 1 al 3} se debe comprender el trabajo y el contexto, 
    describiendo posteriormente el problema que se quiere resolver y los objetivos que se desean 
    lograr. Tras esto, en el estado del arte se analizarán las alternativas existentes en el mercado 
    actualmente en relación con espacios que permitan realizar críticas cinematográficas y compartirlas 
    con los demás usuarios, entendiendo de una mejor manera el dominio del problema.
    \item La segunda parte, construido por el \textbf{capítulo 4}, donde se concentran los matices 
    necesarios para llevar a cabo el proyecto, se planifica el desarrollo, se muestran las distintas herramientas que se usaran en el proyecto y se prepara el terreno para llevar a 
    cabo la implementación de la solución estudiada.
    \item Para terminar, en los \textbf{capítulos 6 y 7} se realizará todo el trabajo de desarrollo y 
    se evaluará la resolución de la propuesta, valorando si ha cumplido con las competencias y 
    objetivos propuestos.
\end{itemize}

A continuación se describirán los objetivos que se desean alcanzar para solucionar el problema que se 
acaba de plantear en este capítulo.

\section{Objetivos}

Tras haber comprendido mejor el problema y lo que supone, se expondrán los objetivos que se pretenden 
lograr a través de la solución a dicho problema.

\begin{itemize}
    \item \textbf{OBJ1} Crear un algoritmo capaz de asegurar la fiabilidad de las reseñas.
    \item \textbf{OBJ2} Crear otro algoritmo que sea capaz de recomendar contenido a los usuarios bajo sus preferencias.
    \item \textbf{OBJ3} Crear una aplicación que contenga ambos algoritmos y puedan ser consumidos por los usuarios.
\end{itemize}

Estos objetivos reflejan las metas generales que se pretenden alcanzar, de esta manera en el último 
capítulo se podrá analizar si realmente la solución dada ha podido lograr dichos objetivos.

\section{Marco de desarrollo}

Un conjunto de metodologías y buenas prácticas de desarrollo en un proyecto de software es esencial 
para establecer un marco de trabajo coherente que guíe la planificación, ejecución y seguimiento de 
todas las actividades. Un conjunto de mejores prácticas y procesos estandarizados que 
ayudan a evitar problemas comunes, aumentar la eficiencia y, en última instancia, asegurar que el 
producto final cumpla con los requisitos y expectativas del cliente. Además, al facilitar la 
comunicación y la colaboración entre los miembros del equipo, las metodologías de desarrollo fomentan 
la cohesión y la productividad del grupo, lo que resulta en una gestión más efectiva del proyecto en 
términos de tiempo y recursos. También, promueve la documentación y la mejora continua.

Debido a la importancia de establecer un marco de desarrollo a seguir para el proyecto, en este caso el marco se
centrará en el enfoque ágil \cite{WhyAgile}. Se presentarán los principios en los que se 
basa el manifiesto ágil \footnote{\url{https://www.agilealliance.org/agile101/the-agile-manifesto/}} y 
las metodologías que se acogen a esos principios, así como la razón detrás de su elección para este 
proyecto. A continuación, se proporcionará una conclusión sobre por qué el enfoque ágil es adecuado y 
beneficioso para este TFG.

Este libro\cite{MetodDesa} es ampliamente respetado en la comunidad de desarrollo ágil y proporciona 
una visión detallada de cómo llevar a cabo estimaciones y planificaciones efectivas en proyectos de 
desarrollo de software utilizando metodologías ágiles. Aunque se centra en la planificación ágil, 
ofrece una comprensión sólida de por qué es esencial adoptar un enfoque estructurado y metodológico en 
el desarrollo de software.

El desarrollo ágil \footnote{\url{https://www.agilealliance.org/agile101/}} es un enfoque metodológico 
utilizado en proyectos de desarrollo de software que se caracteriza por su flexibilidad, adaptabilidad 
y enfoque iterativo e incremental. A diferencia de los enfoques tradicionales, que se basan en planes 
detallados y rigidez en los procesos, el desarrollo ágil se centra en la colaboración, la comunicación 
constante y la capacidad de respuesta a los cambios. Su principal objetivo es entregar un producto de 
calidad que cumpla con las necesidades y expectativas del cliente.

El desarrollo ágil se basa en los siguientes principios:

\begin{itemize}
\item \textbf{Orientación al cliente:} Se busca comprender las necesidades y expectativas del cliente y 
desarrollar soluciones que satisfagan sus requerimientos. La retroalimentación del cliente es 
fundamental para guiar el desarrollo y asegurarse de que el producto final cumpla con sus necesidades.
Por ello se han presentado los usuarios y se han visto reflejadas sus necesidades en las historias de 
usuario, descritas en el segundo capítulo junto al problema. Además de la reflexión vista en la sección 
de motivación en este primer capítulo.

\item \textbf{Enfoque iterativo:} El proyecto se divide en iteraciones cortas y enfocadas, llamadas 
sprints, que tienen una duración definida (por ejemplo, 1 a 4 semanas). Cada sprint tiene objetivos 
claros y entrega un incremento funcional del producto. Al final de cada sprint, se revisa y se ajusta 
el plan en función de la retroalimentación y los resultados obtenidos.
A través de hitos o milestones, usando las herramientas ofrecidas por GitHub descritas más adelante en 
el capítulo de planificación, para definir dichos sprints como milestones, obteniendo tras la 
finalización de cada uno, un producto mínimamente viable (PMV). Analizando y testeando si se obtienen 
los resultados esperados.

\item \textbf{Adaptabilidad y flexibilidad:} El desarrollo ágil reconoce que los requisitos y las 
prioridades pueden cambiar a lo largo del proyecto. En lugar de intentar prever y especificar todos los 
detalles desde el principio, se acepta que algunos requisitos pueden ser ambiguos o desconocidos al 
principio y se permite ajustar y adaptar el plan en función de las necesidades que surjan durante el 
desarrollo.
Esto se ve reflejado en los objetivos principales del proyecto, durante el desarrollo del proyecto 
algunos requisitos han ido cambiando según las necesidades que han surgido. Añadiendo o desechando 
ideas desde que empezó el proyecto.

\item \textbf{Entrega temprana y continua:} El enfoque ágil se enfoca, valga la redundancia, en generar valor para el 
cliente de manera temprana y constante. Se priorizan las funcionalidades más importantes y se entregan en cada 
sprint, lo que permite obtener retroalimentación rápida y garantiza que el producto final cumpla con 
las expectativas del cliente. Para ello se hace uso de \begin{otherlanguage} 
{english}``\textit{\textbf{Pull Request}}''\end{otherlanguage} otra herramienta de GitHub descrita en 
el capítulo de planificación, por la que se tienen entregas continuas y retroalimentación constante. 
Haciendo mucho más fácil y cómoda la resolución de cambios en esas partes del proyecto.

\end{itemize}

El desarrollo ágil se basa en la premisa de que los requerimientos pueden cambiar, los problemas 
pueden surgir y las soluciones pueden evolucionar a lo largo del proyecto. En lugar de resistir a estos 
cambios, el enfoque ágil los abraza y busca manejarlos de manera efectiva a través de la colaboración, 
la iteración y la adaptabilidad.

\subsection{Ventajas del marco elegido}

Las ventajas, por lo general, que ofrece el desarrollo ágil sobre metodologías tradicionales:

\begin{itemize}
\item \textbf{Mayor satisfacción del cliente:} La entrega temprana y continua de incrementos 
funcionales permite obtener retroalimentación del cliente de forma constante, lo que garantiza que el 
producto final cumpla con sus expectativas y necesidades.
\item \textbf{Mayor visibilidad y control del progreso del proyecto:} El enfoque ágil proporciona una 
mayor visibilidad del progreso del proyecto y permite un mayor control sobre el desarrollo 
\cite{VCagileT}, ya que se realizan seguimientos regulares y se realizan ajustes en función de la 
retroalimentación y los resultados obtenidos.
\item \textbf{Mejora en la calidad del producto:} La iteración constante y las pruebas frecuentes 
permiten identificar y corregir rápidamente los problemas, lo que conduce a un producto final de mayor 
calidad.
\end{itemize}

Las ventajas bajo mi experiencia más importantes y destacables, serían la facilidad de adaptarse a los 
cambios a medida que avanzas en el proyecto y la comodidad de los pequeños entregables al usuario de 
forma continua y temprana, facilitando la organización de tareas. Trabajando de manera más simple sobre 
las correcciones dada la continua retroalimentación, aumentando la calidad del trabajo constantemente. 
Además, teniendo en cuenta que estás enfocado en satisfacer al usuario, tienes mayor dominio y 
transparencia del progreso del proyecto. Resumiendo, el enfoque ágil me ha permitido desarrollar de 
manera más efectiva mi TFG, y logrando una gran calidad.

\subsection{Buenas prácticas}

Dentro del marco para el desarrollo software se establecen también unas buenas prácticas para una parte de este, la 
escritura de código. Se guiará mediante el conjunto de principios SOLID\cite{martinFunctional}, conjunto de reglas 
diseñado para la programación orientada a objetos, que asegura la calidad del código, ya que esta está definida por la 
cantidad de principios SOLID que se cumplen en el código. 

El nombre que reciben estos principios es un acrónimo formado por la inicial de los mismos, los cuales fueros creados por \textit{''Robert C. Martin''}\cite{UncleBob} o \begin{otherlanguage} {english}``\textit{\textbf{Uncle Bob}}''\end{otherlanguage}, como se le llama de forma amistosa, famoso ingeniero de software y autor 
estadounidense. Estos principios se crearon para asegurar el sencillo mantenimiento y la escalabilidad de un sistema:

\begin{itemize}
    \item S - (\begin{otherlanguage} {english}\textit{Single Responsibility Principle}):  ``\textit{\textbf{Gather together the things that change for the same reasons. Separate things that change for different reasons}}''\end{otherlanguage},
    la definición de este principio viene a decir que un módulo software solo debería tener una responsabilidad, 
    debería encargarse únicamente de un trabajo y si cambia solo lo debería hacer por una razón. De manera que si un 
    módulo se encarga de diferentes tareas, no solo va a existir una razón de cambiar, sino una por cada tarea 
    diferente de la que se encargue el módulo. 
    \item O - (\begin{otherlanguage} {english}\textit{Open-Closed Principle}): ``\textit{\textbf{A Module should be open for extension but closed for modification}}''\end{otherlanguage},
    este principio advierte que un módulo debe estar abierto a cambios en caso de ampliación, pero cerrado a 
    modificación, si se piensa qué dicho módulo se basa en un inicio de sesión se puede alegar como amplificación el 
    propio inicio de sesión, pero a través de otro sistema, como lo es Google o GitHub. Así se entiende esta 
    amplificación como añadir más funcionalidades al módulo, pero sin perturbar el código ya escrito; en el momento 
    que cualquier funcionalidad añadida haga cambiar el código previo, se consideraría una modificación.
    \item L - (\begin{otherlanguage} {english}\textit{Liskov Substitution Principle}):
    ``\textit{\textbf{A program that uses an interface must not be confused by an implementation of that interface}}''\end{otherlanguage},
    el principio de sustitución indica que no se cometa el error de pensar que las clases que implementan una interfaz
    son implementaciones de esta, cuando realmente son subtipos de dicha interfaz. Este principio previene que al 
    cambiar instancias de una interfaz por subtipos de ella, el funcionamiento del programa no se vea afectado.
    \item I - (\begin{otherlanguage} {english}\textit{Interface Segregation Principle}):
    ``\textit{\textbf{Keep interfaces small so that users don’t end up depending on things they don’t need}}''\end{otherlanguage},
    este hace referencia a que un objeto no debería depender de una interfaz que no usa, o sea que no use todas las
    funciones de dicha interfaz, ya que provoca que crezca el código de la interfaz y se fuerce a objetos a usar 
    funciones que no tienen sentido que utilicen.
    \item D - (\begin{otherlanguage} {english}\textit{Dependency Inversion Principle}):
    ``\textit{\textbf{Depend in the direction of abstraction. High level modules should not depend upon low level details}}''\end{otherlanguage}
    este principio indica que los módulos de alto nivel como la lógica de negocio no dependen de los 
    módulos de bajo nivel como el mecanismo de persistencia, por lo tanto, deben estar separados y ambos deberían 
    depender de abstracciones. Siendo estas lo suficientemente flexibles como para permitir que diferentes
    implementaciones se conecten a ellas, independientemente de los \textit{''detalles''}, implementaciones concretas 
    como, el mecanismo de persistencia o el servicio que se utiliza para acceder a la red. Una técnica que permite
    preservar la inversión de dependencias es la inyección dependencias, esta permite crear la instancia de una clase 
    inyectando como parámetro, en este caso la implementación concreta que se necesite de la abstracción.
\end{itemize}

\subsection{Conclusiones}

Basándose en las características y ventajas expuestas, se ha elegido el enfoque ágil como el enfoque 
principal para el desarrollo de este TFG. La naturaleza iterativa, adaptable y colaborativa del 
desarrollo ágil se alinea de manera efectiva con los objetivos y requisitos del proyecto.

El enfoque ágil permitirá una mayor flexibilidad para adaptarse a posibles cambios en los 
requerimientos, obtener una entrega temprana de funcionalidades y una interacción continua con los usuarios 
finales. Además, facilitará la identificación temprana de problemas y la mejora constante de la calidad 
del producto.

En resumen, el enfoque ágil se considera una elección sólida para este TFG, ya que proporciona una 
estructura flexible y eficiente que permitirá una gestión efectiva del proyecto, una entrega de valor 
continua y una alta probabilidad de alcanzar los objetivos propuestos. Además, de que el conjunto de buenas prácticas 
aseguran la calidad y la prosperidad del código, ofreciendo un sencillo mantenimiento y permitiendo la escalabilidad 
del proyecto.

Este proyecto es software libre, y está liberado con la licencia \cite{gplv3}. Y si se desea se puede encontrar en 
GitHub en este repositorio público \footnote{\url{https://github.com/JoseJordanF/Claqueta}}