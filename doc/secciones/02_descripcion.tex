\chapter{Descripción del problema}

En este capítulo se va a profundizar más en el problema que se plateaba en el primer capítulo.

\section{Problema a resolver}

En la actualidad muchos usuarios disfrutan de visualizar cine, en casa, en el cine o en 
dispositivos móviles. Sin embargo, se suelen enfrentar al problema de decidir que ver y que no, ya que 
hay una ingente cantidad de opciones a su disposición. Aunque existen varias plataformas de reseñas 
cinematográficas, como mencionamos en el capítulo 1, muchas de ellas tienen una cantidad abrumadora de 
reseñas y es complicado determinar cuál es la opción dominante.

Además, algunas de estas plataformas pueden estar sesgadas o influenciadas por factores como la 
publicidad, los intereses corporativos y los prejuicios personales. Como resultado, los usuarios pueden 
no estar obteniendo una visión completa y objetiva de una obra cinematográfica o una serie antes de 
decidir verla.

Por lo que se necesita una aplicación, un espacio de reseñas cinematográficas que ofrezca a los 
usuarios una perspectiva completa y equilibrada de las películas que desean ver. Una 
aplicación que brinde tanto reseñas de usuarios como de críticos de cine profesionales, así como 
calificaciones y datos sobre dicha película o serie. Esto sería muy útil para ayudar a los usuarios a 
tomar decisiones informadas sobre que visualizar.

Resolviendo la base del problema, que los usuarios expresen sus opiniones no es más que el principio, 
ya que esta necesidad viene acompañada de aspectos más personalizables. Lo cual incide en más problemas 
de personalización de dicho espacio, como la facilidad de filtrar las reseñas, según las preferencias 
del usuario, siendo más eficaz en la ayuda de elección de películas. Obtener recomendaciones 
por parte de la aplicación, relacionadas con el contenido que consumes. Y que finalmente este espacio 
logre satisfacer toda necesidad del usuario, disfrutando así del espacio y llegando a promocionar su 
uso a usuarios cercanos a él.

Es muy importante escuchar al cliente, ya que son ellos los que van a explotar el producto, esto se 
verá más profundamente en el capítulo sobre la planificación, donde se hablara de la metodología usada.

Para una buena descripción del problema es esencial describir a las personas afectadas por dicho 
problema. Ya que ellos son los perjudicados, habrá que saber los dispositivos que usan, la 
accesibilidad que tienen a ellos. También las capacidades de estos usuarios y sus habilidades para 
poder definir el ámbito del problema y el propio problema.

Las personas afectadas por este problema que se les plantea cada vez que deciden consumir cualquier
contenido, ya sea para tomar decisiones o compartir su experiencia, serían aquellas a las que les 
gusta, les entretiene o les apasiona el cine u otra obra audiovisual. Y por supuesto le encanta 
compartir su experiencia con quien sea. También formarían parte del problema aquellas personas que 
necesitan de alguna recomendación para consumir cualquier producto audiovisual. Ellos necesitan una 
crítica contundente y objetiva del contenido para llegar a decidir si ver esa obra o no. También a 
estos grupos se unirían aquellos usuarios que les gusta discutir sobre la opinión de los demás acerca 
del contenido audiovisual. Existirán usuarios que pertenezcan a más de uno de estos grupos.

Actualmente, vivimos en una sociedad apegada a las redes sociales, las cuales han permitido la 
comunicación entre usuarios que en su vida jamás habrían podido encontrarse. En sí, el intercambio de 
ideas y el diálogo han sido expandidos a niveles mundiales, prácticamente no existe ninguna barrera
que no te permita comunicarte con quien quieras. La gran mayoría de personas afectadas por el problema 
tienen a su alcance un dispositivo móvil y la inmensa mayoría de usuarios usan como sistema operativo 
móvil ``\textit{Android}'' \cite{AndrvsIOS}. En este grupo encontraremos usuarios con una capacidad 
cognitiva más desarrollada que otros, ya sea por aprendizaje, experiencia o genética. Pero en general 
estos usuarios poseerán ciertas habilidades críticas y analíticas, habilidades argumentativas, 
comunicativas y expresivas. Habilidades ciertamente necesarias para las actividades que realiza este 
grupo de usuarios. Siendo todas estas habilidades fácilmente demostrables a través de  una red social. 
La edad de los usuarios puede variar bastante, ya que se encuentran en un rango muy amplio de edad. El 
mundo de la crítica y las recomendaciones respecto al cine u otros contenidos audiovisuales, es algo 
que le puede fascinar a usuarios de diversas edades, siempre cuando tenga una mínima 
edad para haber desarrollado algunas de las habilidades o capacidades mencionadas anteriormente. En
este rango, la gran mayoría se defenderán en la utilización de un dispositivo móvil, mientras que una 
pequeña parte puede tener problemas con el uso del teléfono móvil.

Gracias a ser una aplicación, los usuarios podrán usar la solución en cualquier lugar, siempre y 
cuando dispongan de conexión a Internet. Su uso generalmente no sería diario, se incrementaría cuando 
el usuario se disponga a ir al cine o ver una serie, siendo su forma de decidir si ver o no ese 
contenido. Tras visualizar el contenido audiovisual, también usaría la aplicación, ya que al usuario le 
gusta expresar su opinión sobre la película o serie que acaba de visualizar. Luego podríamos centrar su 
uso tanto a la previa visualización del contenido como tras su visualización. También podríamos añadir 
un punto de uso durante la visualización de la película o la serie, ya que la solución brinda datos 
relevantes sobre la obra, por lo que el usuario los podría consultar en cualquier momento. La 
aplicación también enviará notificaciones por las interacciones que tengan el resto de usuarios con tus 
reseñas, los usuarios que comentan tus reseñas o los usuarios que indican que les ha gustado o ha sido 
necesaria y relevante para la toma de decisiones de ese usuario. Hablando de las opciones 
personalizables sería una opción avisar de los estrenos de cine a través de notificaciones, 
siendo estas adaptables por el usuario.


\section{Historias de usuario}

Estas historias de usuario reflejan algunos de los problemas y necesidades que podrían abordarse en aplicaciones de reseñas cinematográficas. Cada usuario tiene diferentes perspectivas y prioridades, y las historias de usuario reflejan eso.

\subsection{Aficionado}

Como aficionado al cine, quiero encontrar películas a través de recomendaciones personalizadas basadas en mi consumo de películas.

\subsection{Cinéfilo}

Como cinéfilo, quiero conseguir una visión profesional sobre las películas que consumo, a través de reseñas fiables. Como cinéfilo, necesito participar compartiendo mi experiencia sobre películas con otros cinéfilos para contemplar perspectivas afines o diferentes. 

\subsection{Crítico}

Como crítico de cine, quiero conseguir enseñar a la gente sobre cine a través de mis reseñas. Abriendo sus mentes, impulsando su pensamiento crítico. Para que sepan apreciar el cine y lo disfruten compartiendo sus experiencias.

