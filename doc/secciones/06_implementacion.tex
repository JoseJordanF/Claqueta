\chapter{Implementación}

En este capítulo se va a proceder a desarrollar cada uno de los PMV de cara a cada hito relacionado con 
la implementación, ya que los hitos previos \footnote{\url{https://github.com/JoseJordanF/Claqueta/milestone/1}} \footnote{\url{https://github.com/JoseJordanF/Claqueta/milestone/7}} se encuentran en el capítulo de 
planificación, ya que definen la infraestructura y organización del proyecto. Además, habrá que 
discutir que herramientas van a usarse para desarrollar o llevar a cabo estos hitos y el porqué de su 
elección. 

Dividiéndose la implementación del software en hitos. Estos han sido definidos en Github
y cada uno de ellos contiene un grupo de \textit{issues} que se corresponden con las distintas
mejoras que se han ido incorporando al software a lo largo de su desarrollo.

El uso de herramientas permiten llevar a cabo el proyecto asegurando su calidad y buenas prácticas 
durante el uso de estas.

Por ello se van a describir las herramientas principales que se van a utilizar para el desarrollo del 
software, en cada uno de los hitos. Describiendo los lenguajes de programación que se usaran, lenguajes 
de consulta y manipulación de datos para API, el modelo de datos a usar, a su vez también se mostraran 
herramientas que velen por el buen desarrollo en el repositorio y llevar a cabo buenas prácticas. El 
uso de todas estas herramientas será justificado, explicándose así para qué se va a utilizar dicha 
herramienta y porque se ha elegido.

\section{M1: Definición de objetos - Abstracción del proyecto}

En este hito \footnote{\url{https://github.com/JoseJordanF/Claqueta/milestone/8}} se conseguirá el modelado de los objetos presentes en el problema, para ello se abstraerán 
los conceptos clave y se definirán los objetos de la aplicación. El objetivo es tener una estructura 
clara de los datos a utilizar. La abstracción nos permite identificar las características esenciales, 
eliminando detalles innecesarios. La definición de objetos nos ayudará a comprender sus relaciones, 
atributos y operaciones. Esto establecerá una base sólida para el desarrollo coherente de la aplicación.

