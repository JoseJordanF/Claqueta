\chapter{Implementación}

En este capítulo se va a proceder a desarrollar cada uno de los hitos, obteniendo en un PMV. Pero antes 
habrá que discutir que herramientas van a usarse para desarrollar o llevar a cabo estos hitos. Se 
tomarán decisiones sobre qué herramientas usar y el porqué de su elección. También existirán ciertas herramientas que no se decidan hasta que llegue un hito en el que se deba decidir.

\section{Elección de herramientas para el desarrollo}

Para el desarrollo de un proyecto de ingeniería del software, es indispensable el uso de herramientas 
que permitan llevar a cabo dicho proyecto, asegurando su calidad y buenas prácticas durante el uso de 
estas.

Por ello se van a describir a continuación las herramientas principales que se van a utilizar para el 
desarrollo del software. Describiendo los lenguajes de programación que se usaran, lenguajes de 
consulta y manipulación de datos para API, el modelo de datos a usar, a su vez también se mostraran 
herramientas que velen por el buen desarrollo en el repositorio y llevar a cabo buenas prácticas. El 
uso de todas estas herramientas será justificado, explicándose así para qué se va a utilizar dicha 
herramienta y porque se ha elegido.

\subsection{Lenguaje de programación}

Una de las principales herramientas para el desarrollo de software es el lenguaje de programación, un 
lenguaje que sea afín a las necesidades del proyecto. Siendo así un necesario un lenguaje para modelar 
los objetos descritos en este capítulo, en el primer hito. Pero debemos tener en cuenta que se pretende 
que el desarrollo del modelado de los objetos pueda ser reutilizable, de fácil acceso e intentando 
ahorrar recursos. Por ello, para la aplicación se pretende crear una API propia, como las que se 
mencionaron en el capítulo del estado del arte. 

Por otra parte, también se pretende crear una aplicación que consuma esta API propia. Con lo mencionado 
en el capítulo dos podríamos crear una aplicación Android o una aplicación web, destinada a los 
dispositivos móviles. Según nos menciona \textbf{\textit{Cloud Levante}} en este artículo 
\footnote{\url{https://acortar.link/zynAAq}} lo más usado en móvil son las aplicaciones móviles. Por 
tanto, necesitamos un lenguaje de programación para aplicaciones móviles, más concretamente para 
Android, como mencionamos en el segundo capítulo. Tenemos varias opciones, consultando varios 
blogs\footnote{\url{https://blog.mgpanel.org/post/lenguajes-de-programacion-mas-usados-para-app-moviles}} todos ellos coinciden con qué Java es el primer lenguaje en el que piensas a la hora de crear 
una aplicación para Android. Sin embargo, no es Java el más utilizado, sino una variante de este, 
Kotlin. Debido a su sencillez y facilidad de uso, gracias a la simplicidad de su código será más simple 
crear tus funciones, incluyendo las corrutinas que facilitan la ejecución de tareas asíncronas. Destaca 
su extenso conjunto de librerías y bibliotecas, ya que puede usar las bibliotecas de java debido a la 
interoperabilidad de ambos lenguajes. Mencionar que \textit{Google} ha marcado a Kotlin como lenguaje 
recomendado para aplicaciones Android.

Algo muy importante es que Kotlin es un lenguaje de propósito general, lo cual nos permite crear 
diversos proyectos, entre los que se encuentra el \begin{otherlanguage}
{english}\textit{\textbf{Backend}}\end{otherlanguage} de una aplicación. Pudiendo crear nuestra API 
REST con Kotlin, gracias a diferentes \begin{otherlanguage}
{english}\textit{\textbf{Frameworks}}\end{otherlanguage} \cite{FrameWrk}. Varios de estos módulos de 
\begin{otherlanguage} {english}\textit{software}\end{otherlanguage} nos permiten crear nuestra API 
REST. Entre ellos, el más utilizado era \begin{otherlanguage}
{english}\textbf{\textit{Spring Boot}}\end{otherlanguage}, que está siendo remplazado por otro módulo 
llamado \textbf{\textit{Ktor}} \footnote{\url{https://medium.com/@chaewonkong/ktor-the-next-generation-framework-that-might-replace-spring-boot-868e8d21fc0f}}. Este será el \begin{otherlanguage}
{english}\textit{Framework}\end{otherlanguage} más usado en Kotlin debido a su curva de aprendizaje, su 
alto rendimiento, ser ligero y modular, permitiendo incluir solo los componentes que el desarrollador 
necesita. Y algo relevante, es un \begin{otherlanguage}
{english}\textit{framework}\end{otherlanguage} que se centra en Kotlin, ya que \begin{otherlanguage}
{english}\textit{spring boot}\end{otherlanguage} lo hace en Java y esto lo hace menos idiomático que 
Ktor.

Por lo tanto, al ser un lenguaje de propósito general, que nos permite tanto crear nuestra aplicación 
Android como un \begin{otherlanguage}{english}\textit{backend}\end{otherlanguage}, sea sé nuestra API 
REST, y debido a su popularidad, facilidad de uso y la gran cantidad de recursos que nos ofrece. Kotlin 
es el lenguaje más indicado para el proyecto.

\subsection{Persistencia de datos}

La persistencia de datos es primordial para realizar cualquier lógica de negocio. Necesitamos tanto 
datos del usuario como las entidades que representan los objetos con los que interactúan estos 
usuarios. Para ello necesitamos una base de datos, en Kotlin poseemos una gran variedad de opciones. 
MySQL es ampliamente utilizado y conocido por su rendimiento y confiabilidad. PostgreSQL es otra base 
de datos relacional de código abierto que es altamente capaz y compatible con estándares. SQLite, por 
otro lado, es una base de datos relacional en memoria, ideal para aplicaciones móviles y proyectos más 
pequeños. H2 es una base de datos en memoria que se integra perfectamente con aplicaciones Java y es 
especialmente útil para pruebas y desarrollo local. Por otra parte, es indispensable el uso de una 
herramienta ORM \begin{otherlanguage}
{english}\textit{(Object-Relational Mapping)}\end{otherlanguage} para superar el desfase objeto-
relacional \footnote{\url{https://acortar.link/iDW4Kj}}, ya que pretendemos usar una base de datos 
relacional debido a las relaciones que existen entre las distintas entidades. Por lo que hay varias 
opciones disponibles. Hibernate es un ORM muy popular en el mundo Java, pero su curva de aprendizaje 
puede ser empinada. JPA (Java Persistence API) es una especificación de Java que ofrece múltiples 
implementaciones, como Hibernate y EclipseLink. Ebean es otro ORM conocido por su rendimiento. Sin 
embargo, si estás desarrollando tu aplicación en Kotlin, Exposed 
\footnote{\url{https://github.com/JetBrains/Exposed}}es una excelente elección, ya que está diseñado 
específicamente para Kotlin, aprovechando las características del lenguaje y ofreciendo una sintaxis 
clara y legible. Por lo que sería una buena elección, y teniendo en cuenta que cuando se combina con  
H2 es una excelente opción. Ya que la naturaleza en memoria de este lo hace ideal para pruebas 
unitarias y desarrollo local. La facilidad de configuración y velocidad de ejecución son invaluables en 
entornos de desarrollo ágil. Ya que en el enfoque ágil como sabemos, la adaptabilidad y la capacidad de 
respuesta a los cambios son fundamentales. Por lo tanto, herramientas que se pueden configurar de 
manera eficiente, es decir, ajustar y personalizar según las necesidades cambiantes del proyecto, son 
de gran valor.

\subsection{Testing}

Un proyecto con un enfoque ágil está sujeto a pruebas constantemente, algo que estamos apegando en este 
proyecto a los PMVs resultantes de los milestones. Como ya sabemos de esta manera, aseguramos la 
calidad del producto y nos cerciora de que todo funciona como debería. Consiguiendo así productos de 
calidad más robustos minimizando errores.

Como hemos mencionado, Kotlin goza de acceso a un extenso conjunto de librerías y \textit{frameworks}. 
En este conjunto existen varios \textit{frameworks} que nos permiten testear nuestro código.

Para test unitarios del código encontramos varios \textit{frameworks}, por un lado, tenemos 
\textit{Spek}\footnote{\url{https://github.com/spekframework/spek}}. Una herramienta escrita para 
Kotlin diseñado para facilitar la escritura y ejecución de pruebas en proyectos escritos en este 
lenguaje. Permite definir pruebas en un estilo legible similar al lenguaje natural, lo que facilita su 
comprensión tanto para desarrolladores como para no desarrolladores. Por ello, algunos desarrolladores 
lo relacionan con \begin{otherlanguage}
{english}\textit{Behavior-Driven Development}\end{otherlanguage} (BDD), desarrollo guiado por 
comportamiento. Aunque sus creadores ya han mencionado 
\footnote{\url{https://spekframework.github.io/spek/docs/latest/}} que creen que hay una falsa 
distinción en torno al desarrollo guiado por comportamiento (BDD) y desarrollo guiado por pruebas 
(TDD). Por lo que recomiendan que pensemos en Spek como un simple \textit{framework} de especificación.

También disponemos de la herramienta por defecto que incorpora cualquier tipo de proyecto Kotlin, 
\textit{JUnit5}. Este es la última versión del \textit{framework} de pruebas unitarias para Java. Posee 
una arquitectura modular que se compone de tres módulos principales: JUnit Platform, JUnit Jupiter y 
JUnit Vintage, el primero es el núcleo de la herramienta, el segundo introduce las anotaciones y 
permite configurar los test, y la última permite la compatibilidad con versiones anteriores de este 
\textit{framework}. Este es el más usado actualmente por los desarrolladores Android 
\footnote{\url{https://www.jetbrains.com/es-es/lp/devecosystem-2022/testing/}} como nos indica 
\textbf{\textit{Jetbrains}}, compañía que ha diseñado Kotlin.

Ambos son buenas herramientas de pruebas. Además, permiten la integración con otras bibliotecas o 
\textit{framework} de pruebas. Pero ambas herramientas necesitan de otras bibliotecas imprescindibles 
en las pruebas, estas permiten simular objetos de una clase para trucar el resultado de ciertas 
funciones que queremos testear. Estos objetos se denominan \textbf{\textit{Mock}}, en Kotlin 
encontramos la librería nativa \textit{mockk}. Con el par de uno de los \textit{frameworks} mencionados 
y esta librería podríamos realizar los test unitarios que necesitemos. Pero principalmente si no en su 
totalidad usaremos Junit5 debido a la cantidad de información y ejemplos de uso, además de ser la usada 
por la mayoría de desarrolladores Android.

Para la parte de testeo de UI tenemos acceso a varias herramientas, pero nos limitaremos a usar el 
\textit{framework}\textbf{\textit{Espresso}}\footnote{\url{https://developer.android.com/training/testing/espresso?hl=es-419}}, una herramienta creada por Google y la más recomendada \cite{UITest}.

Por último, necesitamos una herramienta para testear las operaciones que creemos a través de los verbos 
http mencionados antes, para  recuperar, insertar, modificar o eliminar información gracias a nuestra 
API. Como hemos visto entre las herramientas más usadas para las 
pruebas\footnote{\url{https://www.jetbrains.com/es-es/lp/devecosystem-2022/testing/}} se encuentra 
\textit{\textbf{Postman}} que ya hemos mencionado que es una página para ayudar a los desarrolladores 
de API. Por ello es la herramienta que vamos a usar para realizar dichas pruebas, además es sencilla y 
cómoda de usar.





